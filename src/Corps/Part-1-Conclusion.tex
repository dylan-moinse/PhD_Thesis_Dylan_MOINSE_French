%------------------------------%
%% ✎ Dylan (V1) %%%%%%%%% ✅ %%
%% ✎ Alain (V2) %%%%%%%%% ❌ %%
%% ✎ Dylan (V3) %%%%%%%%% ❌ %%
%------------------------------%

% Conclusion de la partie 1
\cleardoublepage
\section*{Conclusion de la partie~I
    \label{part1:conclusion}
    }
    \addcontentsline{toc}{chapter}{Conclusion de la partie~I}

    % Transition
\lettrine[lines=3, findent=8pt, nindent=0pt]{\lettrinefont L}{exploration} des fondements théoriques et méthodologiques menée dans cette première partie a permis d’adosser cette thèse à une réflexion sur les dynamiques de mobilité et territoriales associées au \acrshort{M-TOD} et aux proximités générées par l'essor de la mobilité individuelle légère. Cette démarche a contribué à délimiter le cadre conceptuel dans lequel s’inscrit cette recherche, tout en repérant les exigences méthodologiques déterminantes pour l'investigation des interactions entre transport public et mobilité individuelle légère dans les quartiers de gare, au sein d'un système régional intégré. En posant un socle théorique et méthodologique solide, cette partie constitue un préalable à l'étape empirique qui suivra. Plusieurs enseignements clés en ressortent~: le \acrshort{TOD} doit être réactualisé pour intégrer une réflexion plus approfondie sur les proximités locales et intermodales générées par la mobilité individuelle légère, et sur leur impact sur les formes d’accessibilité~; l’intégration de la mobilité individuelle légère dans le \acrshort{TOD} demeure incomplète, aussi bien dans les travaux académiques que dans les politiques publiques~; l’analyse empirique doit permettre de combler ces lacunes en améliorant les connaissances au sujet des pratiques intermodales effectives et de leurs effets sur les quartiers de gare. Ainsi, cette première partie constitue le socle conceptuel et méthodologique sur lequel repose le reste de la thèse. Elle éclaire les limites du \acrshort{TOD} et ouvre la voie à une étude de cas sur les dynamiques ainsi que sur le potentiel d'usage de la mobilité individuelle légère dans les quartiers de gare. La seconde partie développera cette investigation en mobilisant les outils méthodologiques définis ici, afin d'évaluer l'apport du \acrshort{M-TOD} en termes d'accessibilité.%%Rédigé%%