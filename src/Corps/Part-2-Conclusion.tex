%------------------------------%
%% ✎ Dylan (V1) %%%%%%%%% ✅ %%
%% ✎ Alain (V2) %%%%%%%%% ❌ %%
%% ✎ Dylan (V3) %%%%%%%%% ❌ %%
%------------------------------%

% Conclusion de la partie II
\cleardoublepage
\section*{Conclusion de la partie~II
    \label{part2:conclusion}
    }
    \addcontentsline{toc}{chapter}{Conclusion de la partie~II}

    % Transition
\lettrine[lines=3, findent=8pt, nindent=0pt]{\lettrinefont D}{ivers} enseignements émergent de cette deuxième partie pour repenser le \acrshort{M-TOD}. L’étude des pratiques intermodales souligne que la mobilité individuelle légère constitue un levier stratégique d’accessibilité aux gares, en particulier pour les distances spatio-temporelles comprises entre un et cinq kilomètres, ou à moins de vingt minutes, où les transports en commun ne sont pas compétitifs face à l’automobile. Son développement permet ainsi de renforcer l’attractivité du transport public, en offrant une solution de rabattement et de diffusion efficace et flexible, mieux adaptée aux réalités des déplacements quotidiens. L’analyse spatiale met en évidence que l’intégration de la mobilité individuelle légère redéfinit les périmètres fonctionnels des quartiers de gare, en élargissant les zones et donc les points d'intérêt desservis par le rail et en optimisant la connexion entre les réseaux de transport. Cette évolution invite à une lecture renouvelée du \acrshort{TOD}, où la structuration des territoires ne repose plus uniquement sur la proximité immédiate aux infrastructures ferroviaires, mais intègre une réflexion plus large sur les continuités intermodales. Ces résultats révèlent que les conditions d’intégration de la mobilité individuelle légère restent encore inégales. Ils mettent en avant la nécessité d’une approche plus systémique de l’intermodalité. Ces enseignements préparent ainsi la troisième et dernière partie de cette thèse, qui se dédie à formaliser la conceptualisation d'un \acrshort{M-TOD} sous la forme de propositions de stratégies urbaines. À partir des résultats issus de la modélisation géostatistique, il s’agira de proposer un cadre opérationnel, permettant d'organiser le bon déploiement de l’intégration de la mobilité individuelle légère dans les quartiers de gare et d’identifier les leviers d’action pour une mise en œuvre effective de ce modèle d’urbanisme ferroviaire élargi.%%Rédigé%%