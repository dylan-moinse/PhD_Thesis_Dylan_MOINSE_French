% ___________________________________________
    % Questionnaire FUB
    %\newpage
    \setcounter{section}{0}
\chapterheader{Questionnaire de la Fédération française des Usagers de la Bicyclette}
\chapter{Annexes sur le questionnaire donnant lieu au \textsl{Baromètre des Villes Cyclables}}
    \label{annexes:structure-questionnaire-fub}

    % Renvoi
L'\hyperref[annexes:structure-questionnaire-fub]{annexe~\ref{annexes:structure-questionnaire-fub}} se réfère à la \hyperref[section-chap4:cyclabilite-territoires-genre]{section consacrée à la modélisation de la cyclabilité des territoires en lien avec la pratique genrée de la mobilité individuelle légère} (page \pageref{section-chap4:cyclabilite-territoires-genre}), dans le cadre du \hyperref[chap4:titre]{chapitre~4} (page \pageref{chap4:titre}), et veille à détailler la structure du questionnaire mené par la \textcolor{blue}{\textcite{fub_barometre_2021}}\index{FUB@\textsl{FUB}|pagebf}.%%Rédigé%%

    % ___________________________________________
    % Mini-sommaire
    \setcounter{tocdepth}{2}
    % Redéfinir le titre de la table des matières locale
    \renewcommand{\localcontentsname}{Structure de l'annexe~\ref{annexes:structure-questionnaire-fub}}
\localtableofcontents

    % Description questions
    \newpage
    \needspace{1\baselineskip} % Réserve de l'espace
    \sectionheader{Questions détaillées du questionnaire}
\section{Questions détaillées du questionnaire}
    \label{annexes:structure-questionnaire-fub-questions}

    % Questions détaillées : T1
La première catégorie thématique, identifiée sous l'appellation de \Guillemets{ressenti global} (\(T_{5}\)), explore une série de questions relatives à l'appréciation du vélo (\(Q_{14}\)), à la continuité du réseau cyclable (\(Q_{15}\)), aux éventuels conflits entre cyclistes et piéton·ne·s (\(Q_{16}\)), aux interactions avec les véhicules motorisés (\(Q_{17}\)), à la densité et à la vitesse du trafic automobile (\(Q_{18}\)), ainsi qu'à la démocratisation de l'usage du vélo (\(Q_{19}\)).

    % Questions détaillées : T2
La deuxième section portant sur la \Guillemets{sécurité} (\(T_{2}\)) se concentre sur le sentiment de sécurité pendant la pratique du vélo (\(Q_{20}\)), tant sur les voies principales (\(Q_{21}\)), que dans les rues résidentielles (\(Q_{22}\)), lors des jonctions avec les communes voisines (\(Q_{23}\)) et lors de la traversée des intersections (\(Q_{24}\)), tout en se penchant sur l'inclusivité à l'égard des usager·ère·s vulnérables, à savoir les enfants et les personnes âgées (\(Q_{25}\)).

    % Questions détaillées : T3
Le troisième volet se dédie au \Guillemets{confort} (\(T_{3}\)), à partir du niveau de qualité associé aux itinéraires cyclables (\(Q_{26}\)), à leur entretien (\(Q_{27}\)), à la présence de signalisation routière (\(Q_{28}\)), à la mise en place d'itinéraires temporaires lors de travaux (\(Q_{29}\)) et à la disponibilité de voies cyclables à sens unique (\(Q_{30}\)).

    % Questions détaillées : T4
La quatrième thématique, portant sur les \Guillemets{efforts de la ville} (\(T_{4}\)), examine les actions entreprises par la municipalité pour promouvoir l'usage du vélo (\(Q_{31}\)), les efforts de communication déployés par la ville (\(Q_{32}\)), l'intégration des cyclistes dans les discussions relatives aux projets d'aménagement urbain et de mobilité (\(Q_{33}\)), ainsi que les problématiques liées au stationnement automobile encombrant (\(Q_{34}\)).

    % Questions détaillées : T5
Enfin, la dernière catégorie dédiée aux \Guillemets{services et au stationnement} (\(T_{5}\)) se focalise sur les installations de stationnement pour les vélos en général (\(Q_{35}\)), et plus spécifiquement autour des stations de transport public (\(Q_{36}\)), les services de location de vélos de courte et de longue durée (\(Q_{37}\)), la disponibilité de commerces de vélos et d'ateliers de réparation (\(Q_{38}\)), ainsi que le risque lié aux vols de vélos (\(Q_{39}\)).%%Rédigé%%

    % Tableau questionnaire FUB
    \newpage
    \needspace{1\baselineskip} % Réserve de l'espace
    \sectionheader{Structure du questionnaire}
\section{Structure du questionnaire}
    \label{annexes:structure-questionnaire-fub-tableau}

    % Tableau structure du questionnaire FUB
% Tableau T1
%%Rédigé%%
  \begin{table}[h!]
    \centering
    \renewcommand{\arraystretch}{1.5}
    \resizebox{\columnwidth}{!}{
    \begin{tabular}{p{0.1\columnwidth}p{0.5\columnwidth}p{0.4\columnwidth}}
      % \hline
      \rule{0pt}{15pt} \textcolor{blue}{\textbf{\small{ID}}} & \textcolor{blue}{\textbf{\small{Intitulé de la question}}} & \textcolor{blue}{\textbf{\small{Réponses}}}\\
      \hline
        \multicolumn{3}{l}{\textsl{\textbf{Thématique 1~: Ressenti global}} (\(T_{1}\))}\\
            \hdashline
    \small{\(Q_{14}\)} & \small{\textsl{Se déplacer à vélo dans votre commune est\dots}} & \small{\textsl{Désagréable / Agréable}}\\
        \hdashline
    \small{\(Q_{15}\)} & \small{\textsl{Le réseau d'itinéraires cyclables de ma commune me permet d'aller partout de façon rapide et directe}} & \small{\textsl{Pas du tout / Tout à fait}}\\
        \hdashline
    \small{\(Q_{16}\)} & \small{\textsl{Les conflits entre les personnes circulant à vélo et à pied sont\dots}} & \small{\textsl{Très fréquents / Très rares}}\\
        \hdashline
    \small{\(Q_{17}\)} & \small{\textsl{À vélo, les personnes conduisant des véhicules motorisés me respectent}} & \small{\textsl{Pas du tout / Tout à fait}}\\
        \hdashline
    \small{\(Q_{18}\)} & \small{\textsl{À vélo, je trouve que le trafic motorisé (volume et vitesse) est\dots}} & \small{\textsl{Insupportable / Pas du tout gênant}}\\
        \hdashline
    \small{\(Q_{19}\)} & \small{\textsl{Selon moi, dans ma commune, l'usage du vélo est\dots}} & \small{\textsl{Limité à certains / Très démocratisé}}\\
        \hline
    \end{tabular}}
    \caption*{}
    \vspace{5pt}
        \begin{flushright}\scriptsize
        Source~: \textcolor{blue}{\textcite{fub_barometre_2021}}
        \end{flushright}
        \end{table}

% Tableau T2
%%Rédigé%%
  \begin{table}[h!]
    \centering
    \renewcommand{\arraystretch}{1.5}
    \resizebox{\columnwidth}{!}{
    \begin{tabular}{p{0.1\columnwidth}p{0.5\columnwidth}p{0.4\columnwidth}}
      % \hline
      \rule{0pt}{15pt} \textcolor{blue}{\textbf{\small{ID}}} & \textcolor{blue}{\textbf{\small{Intitulé de la question}}} & \textcolor{blue}{\textbf{\small{Réponses}}}\\
      \hline
        \multicolumn{3}{l}{\textsl{\textbf{Thématique 2~: Sécurité}} (\(T_{2}\))}\\
            \hdashline
    \small{\(Q_{20}\)} & \small{\textsl{En général, quand je circule à vélo dans ma commune, je me sens\dots}} & \small{\textsl{En sécurité / En danger}}\\
        \hdashline
    \small{\(Q_{21}\)} & \small{\textsl{Je peux circuler à vélo en sécurité sur les grands axes dans ma commune}} & \small{\textsl{Pas du tout / Tout à fait}}\\
        \hdashline
    \small{\(Q_{22}\)} & \small{\textsl{Je peux circuler à vélo en sécurité dans les rues résidentielles}} & \small{\textsl{Pas du tout / Tout à fait}}\\
        \hdashline
    \small{\(Q_{23}\)} & \small{\textsl{Je peux rejoindre à vélo en sécurité les communes voisines}} & \small{\textsl{Pas du tout / Tout à fait}}\\
        \hdashline
    \small{\(Q_{24}\)} & \small{\textsl{Selon moi, traverser un carrefour ou un rond-point est\dots}} & \small{\textsl{Toujours dangereux / Jamais dangereux}}\\
        \hdashline
    \small{\(Q_{25}\)} & \small{\textsl{Pour les enfants et les personnes âgées, circuler à vélo est\dots}} & \small{\textsl{Très dangereux / Très sûr}}\\
        \hline
    \end{tabular}}
    \caption*{}
    \vspace{5pt}
        \begin{flushright}\scriptsize
        Source~: \textcolor{blue}{\textcite{fub_barometre_2021}}
        \end{flushright}
        \end{table}

% Tableau T3
%%Rédigé%%
  \begin{table}[h!]
    \centering
    \renewcommand{\arraystretch}{1.5}
    \resizebox{\columnwidth}{!}{
    \begin{tabular}{p{0.1\columnwidth}p{0.5\columnwidth}p{0.4\columnwidth}}
      % \hline
      \rule{0pt}{15pt} \textcolor{blue}{\textbf{\small{ID}}} & \textcolor{blue}{\textbf{\small{Intitulé de la question}}} & \textcolor{blue}{\textbf{\small{Réponses}}}\\
      \hline
        \multicolumn{3}{l}{\textsl{\textbf{Thématique 3~: Confort}} (\(T_{3}\))}\\
            \hdashline
    \small{\(Q_{26}\)} & \small{\textsl{Selon moi, les itinéraires cyclables sont\dots}} & \small{\textsl{Pas du tout confortables / Très confortables}}\\
        \hdashline
    \small{\(Q_{27}\)} & \small{\textsl{L'entretien des itinéraires cyclables est\dots}} & \small{\textsl{Très mauvais / Très bon}}\\
        \hdashline
    \small{\(Q_{28}\)} & \small{\textsl{Les directions à vélo sont correctement indiquées par des panneaux}} & \small{\textsl{Pas du tout / Tout à fait}}\\
        \hdashline
    \small{\(Q_{29}\)} & \small{\textsl{Lors de travaux sur les itinéraires cyclables, une solution alternative sûre est proposée}} & \small{\textsl{Jamais / Toujours}}\\
        \hdashline
    \small{\(Q_{30}\)} & \small{\textsl{À vélo, je suis autorisé à emprunter les voies à sens unique à contre-sens}} & \small{\textsl{Jamais / Toujours}}\\
        \hline
    \end{tabular}}
    \caption*{}
    \vspace{5pt}
        \begin{flushright}\scriptsize
        Source~: \textcolor{blue}{\textcite{fub_barometre_2021}}
        \end{flushright}
        \end{table}

% Tableau T4
%%Rédigé%%
  \begin{table}[h!]
    \centering
    \renewcommand{\arraystretch}{1.5}
    \resizebox{\columnwidth}{!}{
    \begin{tabular}{p{0.1\columnwidth}p{0.5\columnwidth}p{0.4\columnwidth}}
      % \hline
      \rule{0pt}{15pt} \textcolor{blue}{\textbf{\small{ID}}} & \textcolor{blue}{\textbf{\small{Intitulé de la question}}} & \textcolor{blue}{\textbf{\small{Réponses}}}\\
      \hline
        \multicolumn{3}{l}{\textsl{\textbf{Thématique 4~: Efforts de la commune}} (\(T_{4}\))}\\
            \hdashline
    \small{\(Q_{31}\)} & \small{\textsl{Selon moi, les efforts faits en faveur du vélo par la ville sont\dots}} & \small{\textsl{Inexistants / Importants}}\\
        \hdashline
    \small{\(Q_{32}\)} & \small{\textsl{La communication en faveur des déplacements à vélo est\dots}} & \small{\textsl{Inexistante / Importante}}\\
        \hdashline
    \small{\(Q_{33}\)} & \small{\textsl{La mairie est à l'écoute des besoins des usagers du vélo, elle les implique dans ses réflexions sur les mobilités et les projets d'aménagement}} & \small{\textsl{Jamais / Toujours}}\\
        \hdashline
    \small{\(Q_{34}\)} & \small{\textsl{Selon moi, le stationnement des véhicules motorisés (voitures, camions, motos\dots) sur les itinéraires cyclables est\dots}} & \small{\textsl{Très fréquent / Très rares}}\\
        \hline
    \end{tabular}}
    \caption*{}
    \vspace{5pt}
        \begin{flushright}\scriptsize
        Source~: \textcolor{blue}{\textcite{fub_barometre_2021}}
        \end{flushright}
        \end{table}

% Tableau T5
%%Rédigé%%
  \begin{table}[h!]
    \centering
    \renewcommand{\arraystretch}{1.5}
    \resizebox{\columnwidth}{!}{
    \begin{tabular}{p{0.1\columnwidth}p{0.5\columnwidth}p{0.4\columnwidth}}
      % \hline
      \rule{0pt}{15pt} \textcolor{blue}{\textbf{\small{ID}}} & \textcolor{blue}{\textbf{\small{Intitulé de la question}}} & \textcolor{blue}{\textbf{\small{Réponses}}}\\
      \hline
        \multicolumn{3}{l}{\textsl{\textbf{Thématique 5~: Services et stationnements}} (\(T_{5}\))}\\ 
            \hdashline
    \small{\(Q_{35}\)} & \small{\textsl{Dans la commune ou à proximité, trouver un stationnement vélo adapté à mon besoin (durée, sécurité\dots) est\dots}} & \small{\textsl{Impossible / Très facile}}\\
        \hdashline
    \small{\(Q_{36}\)} & \small{\textsl{Stationner son vélo en gare ou à une station de transports en commun est\dots}} & \small{\textsl{Impossible / Très facile}}\\
        \hdashline
    \small{\(Q_{37}\)} & \small{\textsl{Louer un vélo pour quelques heures ou pour plusieurs mois est\dots}} & \small{\textsl{Impossible / Très facile}}\\
        \hdashline
    \small{\(Q_{38}\)} & \small{\textsl{Dans la commune ou à proximité, trouver un magasin ou un atelier de réparation est\dots}} & \small{\textsl{Impossible / Très facile}}\\
        \hdashline
    \small{\(Q_{39}\)} & \small{\textsl{Selon moi, les vols de vélos sont\dots}} & \small{\textsl{Très fréquents / Très rares}}\\
      \hline
    \end{tabular}}
    \caption*{}
    \vspace{5pt}
        \begin{flushleft}\scriptsize
        \textcolor{blue}{Lecture~:} dans le \textsl{Baromètre des Villes Cyclables}, 26 questions (de \(Q_{14}\) à \(Q_{39}\)) servent à déterminer la notation moyenne de cyclabilité des communes, au travers de 5 thématiques (de \(T_{1}\) à \(T_{5}\)).
        \end{flushleft}
        \begin{flushright}\scriptsize
        Source~: \textcolor{blue}{\textcite{fub_barometre_2021}}
        \end{flushright}
        \end{table}%%Rédigé%%