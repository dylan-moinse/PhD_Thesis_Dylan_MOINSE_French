% ___________________________________________
    % RSL
    %\newpage
    \setcounter{section}{0}
\chapterheader{Annexes sur la revue systématique de la littérature}
\chapter{Annexes sur la revue systématique de la littérature sur un urbanisme orienté vers les transports en commun et soutenu par la mobilité individuelle légère}
    \label{annexes:rsl}

    % Renvoi
L'\hyperref[annexes:rsl]{annexe~\ref{annexes:rsl}} se réfère au \hyperref[chap2:titre]{chapitre~2} (page \pageref{chap2:titre}), et veille à donner une vue d'ensemble sur les productions scientifiques et techniques rassemblées dans la \acrfull{RSL} sur un \acrfull{B-TOD} et un \acrfull{M-TOD}.%%Rédigé%%

    % ___________________________________________
    % Mini-sommaire
    \setcounter{tocdepth}{2}
    % Redéfinir le titre de la table des matières locale
    \renewcommand{\localcontentsname}{Structure de l'annexe~\ref{annexes:rsl}}
\localtableofcontents

    % Tableau types de combinaison modale
    \newpage
    \needspace{1\baselineskip} % Réserve de l'espace
    \sectionheader{Combinaisons modales étudiées}
\section{Combinaisons modales étudiées}
    \label{annexes:rsl-combinaisons-modales}

    % Tableau combinaisons RSL
% Types de combinaison modale
%%Rédigé%%
  \begin{table}[h!]
    \centering
    \renewcommand{\arraystretch}{1.5}
    \resizebox{\columnwidth}{!}{
    \begin{tabular}{p{0.6\columnwidth}p{0.2\columnwidth}p{0.2\columnwidth}}
      % \hline
      \rule{0pt}{15pt} \textcolor{blue}{\textbf{\small{Type de combinaison modale étudié}}} & \textcolor{blue}{\textbf{\small{Effectif}}} & \textcolor{blue}{\textbf{\small{Part}}}\\
      \hline
\multicolumn{3}{l}{\textbf{Corpus impliquant le vélo classique}}\\
    \small{Train} & \small{44} & \small{40~\%}\\
    \small{Au moins deux systèmes de transport en commun} & \small{34} & \small{31~\%}\\
    \small{Bus à haut niveau de service ou bus} & \small{16} & \small{15~\%}\\
    \small{Métro ou tramway} & \small{15} & \small{14~\%}\\
\hdashline
\multicolumn{3}{l}{\textbf{Corpus impliquant le vélo en libre-service avec station}}\\
    \small{Métro ou tramway} & \small{31} & \small{54~\%}\\
    \small{Au moins deux systèmes de transport en commun} & \small{21} & \small{37~\%}\\
    \small{Bus à haut niveau de service ou bus} & \small{4} & \small{7~\%}\\
    \small{Train} & \small{1} & \small{2~\%}\\
\hdashline
\multicolumn{3}{l}{\textbf{Corpus impliquant le vélo en libre-service sans station}}\\
    \small{Métro ou tramway} & \small{27} & \small{73~\%}\\
    \small{Au moins deux systèmes de transport en commun} & \small{6} & \small{16~\%}\\
    \small{Bus à haut niveau de service ou bus} & \small{2} & \small{5~\%}\\
    \small{Train} & \small{2} & \small{5~\%}\\
\hdashline
\multicolumn{3}{l}{\textbf{Corpus impliquant la trottinette électrique en libre-service sans station}}\\
    \small{Au moins deux systèmes de transport en commun} & \small{12} & \small{57~\%}\\
    \small{Métro ou tramway} & \small{5} & \small{24~\%}\\
    \small{Bus à haut niveau de service ou bus} & \small{3} & \small{14~\%}\\
    \small{Train} & \small{1} & \small{5~\%}\\
        \hline
    \end{tabular}}
    \caption*{}
    \vspace{5pt}
        \begin{flushright}\scriptsize
        Auteur~: \textcolor{blue}{Dylan Moinse (2023)}
        \end{flushright}
        \end{table}

    % Tableau résultats
    \needspace{1\baselineskip} % Réserve de l'espace
    \sectionheader{Dimensions traitées}
\section{Dimensions traitées}
    \label{annexes:rsl-resultats}

    % Tableau résultats RSL
% Résultats
%%Rédigé%%
  \begin{table}[h!]
    \centering
    \renewcommand{\arraystretch}{1.5}
    \resizebox{\columnwidth}{!}{
    \begin{tabular}{p{0.6\columnwidth}p{0.2\columnwidth}p{0.2\columnwidth}}
      % \hline
      \rule{0pt}{15pt} \textcolor{blue}{\textbf{\small{Aspect considéré}}} & \textcolor{blue}{\textbf{\small{Effectif}}} & \textcolor{blue}{\textbf{\small{Part}}}\\
      \hline
    \small{Densité de population et des emplois (\(D1\))} & \small{44} & \small{18~\%}\\
    \small{Diversité urbaine (\(D2\))} & \small{36} & \small{15~\%}\\
    \small{\textsl{Design} (\(D3\))} & \small{78} & \small{33~\%}\\
    \small{Accessibilité à la destination (\(D4\))} & \small{61} & \small{26~\%}\\
    \small{Distance aux réseaux de transport en commun (\(D5\))} & \small{121} & \small{51~\%}\\
    \small{Management de la demande (\(D6\))} & \small{90} & \small{38~\%}\\
    \small{Caractéristiques socio-démographiques (\(D7\))} & \small{48} & \small{20~\%}\\
    \small{Impacts} & \small{84} & \small{35~\%}\\
    \small{Gouvernance} & \small{13} & \small{5~\%}\\
        \hline
    \end{tabular}}
    \caption*{}
    \vspace{5pt}
        \begin{flushright}\scriptsize
        Auteur~: \textcolor{blue}{Dylan Moinse (2023)}
        \end{flushright}
        \end{table}