% ___________________________________________
    % Détours angles
    %\newpage
    \setcounter{section}{0}
\chapterheader{Notes complémentaires sur la mesure des angles}
\chapter{Annexes sur la détermination des angles dans la géométrie des détours}
    \label{annexes:calcul-detours}

    % Renvoi
L'\hyperref[annexes:calcul-detours]{annexe~\ref{annexes:calcul-detours}} se réfère à la \hyperref[chap5:detours-pauses-optimisation]{section consacrée aux détours et aux pauses} (page \pageref{chap5:detours-pauses-optimisation}), dans le cadre du \hyperref[chap5:titre]{chapitre~5} (page \pageref{chap5:titre}), et veille à détailler les calculs réalisés afin de déterminer les angles des déplacements intermodaux comportant des détours géométriques.%%Rédigé%%

    % ___________________________________________
    % Mini-sommaire
    \setcounter{tocdepth}{2}
    % Redéfinir le titre de la table des matières locale
    \renewcommand{\localcontentsname}{Structure de l'annexe~\ref{annexes:calcul-detours}}
\localtableofcontents

    \newpage
    % Structure questionnaire
    \newpage
    \needspace{1\baselineskip} % Réserve de l'espace
    \sectionheader{Calcul des angles}
\section{Calcul des angles}
    \label{annexes:calcul-detours-angles}

    % Coordonnées cartésiennes
\subsection{Coordonnées cartésiennes}
    \label{annexes:calcul-detours-angles-cartesien}

    % Coordonnées cartésiennes
En nous appuyant sur les coordonnées géographiques de chaque point \(A\), \(B\) et \(C\), exprimées en latitude et longitude en utilisant le système de référence \acrshort{WGS84}, nous les avons converties en coordonnées cartésiennes (\(x\), \(y\), \(z\)) en utilisant les formules de conversion pour une sphère.%%Rédigé%%

    \begin{equation*}
    \begin{array}{lclclclclcl}
    \displaystyle x_{(A, B, C)} &=& R* cos(latitude) * cos(longitude)\\\\
    \displaystyle y_{(A, B, C)} &=& R* cos(latitude) * sin(longitude)\\\\
    \displaystyle z_{(A, B, C)} &=& R* sin(latitude)\\\\
    \end{array}
    \end{equation*}

    \begin{align*}
    &\text{où~:} \\
    &R \text{ représente le rayon de la Terre.}
    \end{align*}

    % Vecteurs AB et BC
\subsection{Vecteurs}
    \label{annexes:calcul-detours-angles-vecteurs}

    % Vecteurs AB et BC
À l'aide des coordonnées cartésiennes déterminées \(x_{(A, B, C)}\), \(y_{(A, B, C)}\) et \(z_{(A, B, C)}\), nous avons ensuite calculé les vecteurs \(AB\) et \(BC\) en soustrayant les coordonnées cartésiennes des points \(A\) et \(B\), puis \(B\) et \(C\).%%Rédigé%%

    \begin{equation*}
    \begin{array}{lclclclclcl}
    \displaystyle AB &=& (x_B - x_A, y_B - y_A, z_B - z_A)\\\\
    \displaystyle BC &=& (x_C - x_B, y_C - y_B, z_C - z_B)\\\\
    \end{array}
    \end{equation*}

    % Produit scalaire et norme des vecteurs
\subsection{Produit scalaire et norme des vecteurs}
    \label{annexes:calcul-detours-angles-produit-scalaire}

    % Produit scalaire et norme des vecteurs
À partir des vecteurs \(AB\) et \(BC\) obtenus, nous avons calculé respectivement leur produit scalaire (\(AB \cdot BC\)) et leur norme (\(||{AB}||\)) et (\(||{BC}||\)).%%Rédigé%%

    \begin{equation*}
    \begin{array}{lclclclclcl}
    \displaystyle AB \cdot BC &=& (x_{AB} * x_{BC}) + (y_{AB} * y_{BC}) + (z_{AB} * z_{BC})\\
    \end{array}
    \end{equation*}

    \begin{align*}
    &\text{où~:} \\
    &x/y/z_{AB} \text{ représentent les composantes du vecteur AB ;}\\
    &x/y/z_{BC} \text{ représentent les composantes du vecteur BC.}
    \end{align*}

    \begin{equation*}
    \begin{array}{lclclclclcl}
    \displaystyle ||AB|| &=& \sqrt{(x_{AB}^2 + y_{AB}^2 + z_{AB}^2)}\\
    \displaystyle ||BC|| &=& \sqrt{(x_{BC}^2 + y_{BC}^2 + z_{BC}^2)}\\
    \end{array}
    \end{equation*}

    % Angle
\subsection{Détermination des angles en radians et en degrés}
    \label{annexes:calcul-detours-angles-produit-angle}

    % Angle
La dernière étape a consisté à mesurer l'angle du point \(B\) entre les vecteurs \(AB\) et \(BC\), exprimé en radians \(\delta\) et en degrés \(\alpha\).%%Rédigé%%

    \begin{equation*}
    \begin{array}{lclcl}
    \displaystyle \delta &=& \displaystyle\frac{AB \cdot BC}{||AB|| * ||BC||}\\\\
    \end{array}
    \end{equation*}

    \begin{align*}
    &\text{où~:} \\
    &acos \text{ représente la fonction $arccosinus$.}\\
    \end{align*}

    \begin{equation*}
    \begin{array}{lclclclclcl}
    \displaystyle \alpha &=& \delta * (\displaystyle\frac{180}{\pi})\\\\
    \end{array}
    \end{equation*}