%------------------------------%
%% ✎ Dylan (V1) %%%%%%%%% ✅ %%
%% ✎ Alain (V2) %%%%%%%%% ❌ %%
%% ✎ Dylan (V3) %%%%%%%%% ❌ %%
%------------------------------%

% Conclusion de la partie III
\cleardoublepage
\section*{Conclusion de la partie~III
    \label{part3:conclusion}
    }
    \addcontentsline{toc}{chapter}{Conclusion de la partie~III}

    % Transition
\lettrine[lines=3, findent=8pt, nindent=0pt]{\lettrinefont À}{} travers cette troisième et dernière partie, nous avons proposé une formalisation du \acrshort{M-TOD}, visant à confronter cette déclinaison du \acrshort{TOD} aux réalités urbaines et sociales du territoire étudié. En structurant ces apports autour d’une réflexion sur les infrastructures, les services de mobilité et les configurations territoriales des quartiers de gare, cette formalisation contribue à renouveler la lecture de l’urbanisme ferroviaire et des stratégies d’aménagement des territoires desservis par le rail. Cette dernière partie constitue ainsi l’aboutissement du travail mené dans cette thèse, en traduisant les constats théoriques et empiriques en une proposition structurée et opérationnelle d’urbanisme ferroviaire intermodal. En intégrant pleinement la mobilité individuelle légère dans la conception des quartiers de gare et des pôles d’échange, le \acrshort{M-TOD} ouvre une nouvelle perspective sur la planification des mobilités urbaines. Il dépasse une vision segmentée des modes de déplacement pour promouvoir un système intermodal performant en articulation avec le système urbain. Ces réflexions s’inscrivent dans un débat plus large sur l’évolution du \acrshort{TOD}, dans un contexte marqué par la transition écologique et la transformation des mobilités urbaines. Elles apportent également des éléments concrets pour les acteurs urbains, souhaitant intégrer la mobilité individuelle légère dans leurs stratégies d’aménagement et de transport public. L’ensemble des résultats tirés de cette recherche doctorale permet ainsi d’affirmer que le \acrshort{M-TOD} constitue une réponse pertinente aux défis de l’accessibilité intermodale et de la structuration des quartiers de gare. Ces éléments feront l’objet de la conclusion générale de cette thèse, qui reviendra sur les principaux apports de ce travail, ses implications scientifiques et techniques, ainsi que sur les perspectives de recherche et d’action à envisager pour accompagner la transition vers un urbanisme ferroviaire pleinement intégré à ces pratiques de mobilité émergentes.%%Rédigé%%