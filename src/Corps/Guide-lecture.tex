%------------------------------%
%% ✎ Dylan (V1) %%%%%%%%% ✅ %%
%% ✎ Alain (V2) %%%%%%%%% ❌ %%
%% ✎ Dylan (V3) %%%%%%%%% ❌ %%
%------------------------------%

%\cleardoublepage
    \needspace{1\baselineskip} % Réserve de l'espace
\chapter*{Clés de lecture
    \label{body:guide-lecture-manuscrit}
    }
    \markboth{Guide de lecture du manuscrit}{}
    \markright{Préambule}{}
    \addcontentsline{toc}{part}{Guide de lecture du manuscrit}

% --- %
    % Ecriture inclusive
\section*{Adoption de l'écriture inclusive
    \label{subbody:adoption-ecriture-inclusive}
    }
    \addcontentsline{toc}{section}{Adoption de l'écriture inclusive}

\lettrine[lines=3, findent=8pt, nindent=0pt]{\lettrinefont A}{dopter} une approche d'écriture inclusive vise à promouvoir une représentation équilibrée des genres dans la langue écrite, afin de lutter contre les stéréotypes de genre tout en mettant en valeur la diversité des personnes. L'écriture inclusive ne prétend pas complexifier la langue, mais plutôt rendre l'écriture plus représentative de la réalité sociale en nous invitant à repenser notre façon de communiquer. Au lieu d'alourdir la lecture, l'écriture inclusive offre bien au contraire des tournures grammaticales plus fluides et concises. D'après le manuel d'écriture inclusive publié par \textcolor{blue}{Raphaël} \textcolor{blue}{\textcite[19-23]{haddad_manuel_2019}}\index{Haddad, Raphaël|pagebf}, l'écriture inclusive permet de déconstruire dix arguments majeurs~: (i) elle n'aurait pas d'influence sur nos représentations~; (ii) serait trop compliquée à utiliser~; (iii) encombrerait, et même (iv) défigurerait le texte~; (v) elle serait interdite par certaines institutions, (vi) car elle menacerait la langue française, (vii) en promouvant la \textsl{novlangue}, (viii) et en ayant des règles hétérogènes~; (ix) tandis que le masculin serait le marqueur du neutre, (x) et qu'il serait signe de prestige en français.%%Rédigé%%

À cet égard, nous avons fait le choix d'inclure le point médian (·) qui intègre à la fois la forme féminine et masculine. Conscient·e·s que cette démarche nécessite un temps d'adaptation, nous avons fait le choix d'énumérer les quelques mots spécifiquement affectés par l'adoption de l'écriture inclusive afin de guider la lecture pour tou·te·s.%%Rédigé%%

% --- %
    % Types de productions scientifiques
    \needspace{1\baselineskip} % Réserve de l'espace
\section*{Nomenclature des productions scientifiques
    \label{subbody:nomenclature-productions-scientifiques}
    }
    \addcontentsline{toc}{section}{Nomenclature des productions scientifiques en Sciences Humaines et Sociales}

Le \acrfull{Hcéres} a établi, en France, une nomenclature des productions scientifiques en \acrfull{SHS}. Ce référentiel vise à catégoriser les différents types de productions scientifiques réalisées par les chercheur·se·s, dans le cadre de l'évaluation des structures de recherche. La codification proposée est présente tout au long de la thèse de doctorat, au sein des encadrés listant les productions scientifiques en lien avec chaque chapitre. La nomenclature est la suivante \textcolor{blue}{\autocite{ministere_de_leducation_nationale_de_lenseignement_superieur_et_de_la_recherche_nomenclatures_nodate}}\index{Ministère de l'Éducation Nationale, de l'Enseignement Supérieur et de la Recherche@\textsl{Ministère de l'Éducation Nationale, de l'Enseignement Supérieur et de la Recherche}|pagebf}~:%%Rédigé%%

    \needspace{1\baselineskip} % Réserve de l'espace
\subsubsection*{Publications scientifiques~:}
    \begin{customitemize}
\item \textbf{ACL}~: Articles dans des revues internationales ou nationales avec comité de lecture répertoriées par l'Hcéres ou dans les bases de données internationales~;
\item \textbf{ACLN}~: Articles dans des revues avec comité de lecture non répertoriées par l'Hcéres ou dans des bases de données internationales~;
\item \textbf{ASCL}~: Articles dans des revues sans comité de lecture~;
\item \textbf{OS}~: Ouvrages scientifiques (y compris les éditions critiques et les traductions scientifiques)~;
\item \textbf{PT}~: Publications de transfert.
    \end{customitemize}%%Rédigé%%

    \needspace{1\baselineskip} % Réserve de l'espace
\subsubsection*{Manifestations scientifiques~:}
    \begin{customitemize}
\item \textbf{C-INV}~: Conférences données à l'invitation du Comité d'organisation dans un congrès national ou international~;
\item \textbf{C-ACTI}~: Communications avec actes dans un congrès international~;
\item \textbf{C-ACTN}~: Communications avec actes dans un congrès national~;
\item \textbf{C-COM}~: Communications orales sans actes dans un congrès international ou national~;
\item \textbf{C-AFF}~: Communications par affiche dans un congrès international ou national.
    \end{customitemize}%%Rédigé%%

    \needspace{1\baselineskip} % Réserve de l'espace
\subsubsection*{Diffusions de la culture scientifique~:}
    \begin{customitemize}
\item \textbf{PV}~: Publications de vulgarisation~;
\item \textbf{PAT}~: Productions artistiques théorisées.
    \end{customitemize}%%Rédigé%%

    \needspace{1\baselineskip} % Réserve de l'espace
\subsubsection*{Autres productions~:}
    \begin{customitemize}
\item \textbf{BRE}~: Brevets~;
\item \textbf{DO}~: Directions d'ouvrages ou de revues~;
\item \textbf{OR}~: Outils de recherche~;
\item \textbf{TH}~: Thèses de doctorat~;
\item \textbf{AP}~: Autres productions.
    \end{customitemize}%%Rédigé%%

% --- %
    % LaTeX
    \needspace{1\baselineskip} % Réserve de l'espace
\section*{Intérêt de \latexword{\LaTeX} en \acrlong{SHS}
    \label{subbody:interet-latex}
    }
    \addcontentsline{toc}{section}{Intérêt de \latexword{\LaTeX} en Sciences Humaines et Sociales}

    % Rédaction en LaTeX
En clôture de cette partie dédiée aux recommandations de lecture de ce manuscrit, il est opportun de porter à la connaissance du·de la lecteur·rice que la présente thèse de doctorat a été rédigée au moyen du système de composition de documents \latexword{\LaTeX}. La décision d'opter pour ce langage informatique, dit de \Guillemets{balisage léger} \textcolor{blue}{\autocite[16]{pochet_markdown_2023}}\index{Pochet, Bernard|pagebf}, ne saurait rester injustifiée, tant le bref exposé de ses avantages peut, nous l'espérons, encourager son adoption dans nos disciplines. En effet, l'utilisation de \latexword{\LaTeX}, bien au-delà d'une simple préférence stylistique, s'inscrit dans une volonté d'excellence académique et de renouvellement des pratiques de recherche. Pourtant, bien que les ouvrages en \latexword{\LaTeX} soient très répandus, il n'empêche que peu d'entre eux sont spécifiquement consacrés aux \acrshort{SHS} \textcolor{blue}{\autocite[7]{rouquette_xelatex_2012}}\index{Rouquette, Maïeul|pagebf}.%%Rédigé%%

    % Avantages de LaTeX
En reprenant l'argumentation développée par \textcolor{blue}{\textcite[8-9]{rouquette_xelatex_2012}}\index{Rouquette, Maïeul|pagebf} dans son ouvrage intitulé \textsl{(Xe)LaTeX appliqué aux sciences humaines}, il apparaît que les logiciels de type \Guillemets{ce que vous voyez est ce que vous obtenez}, ou \acrfull{WYSIWYG}, tels que \Marque{Microsoft Office Word}\footnote{
\Marque{Microsoft Office Word} (\url{https://www.microsoft.com/fr-fr/microsoft-365/word}) est un logiciel de traitement de texte, distribué depuis 1983 et est aujourd'hui intégré à la suite \textsl{Microsoft Office}.
} ou \Marque{LibreOffice}\footnote{
\Marque{LibreOffice} (\url{https://www.libreoffice.org/}) est une suite bureautique libre et gratuite, développée par l'organisation à but non lucratif \Marque{Document Foundation}.
}, rencontrent divers obstacles dans le traitement de texte, particulièrement pour les longs documents. À l'opposé, le système et le langage \latexword{\LaTeX} se distingue par sa qualité de précision typographique, facilitant la gestion des documents volumineux et la manipulation des références bibliographiques. La séparation entre l'éditeur de texte et le compilateur en \latexword{\LaTeX} permet une distinction claire entre le contenu et la mise en forme du document, produit en format ouvert PDF (\textsl{Portable Document Format}). De notre point de vue, l'utilisation de \latexword{\LaTeX} se justifie par de multiples avantages, dans le cadre de nos recherches doctorales~:
    \begin{customitemize}
\item Reproductibilité et pérennité du manuscrit, indépendamment des évolutions logicielles~;
\item Langage informatique libre, gratuit, fiable en existant depuis plus de trois décennies et compatible avec tous les systèmes d'exploitation~;
\item Facilité d'accès grâce au format PDF~;
\item Gestion optimisée et gain de temps sur les longs documents, comme pour une thèse de doctorat~;
\item Conformité aux normes des éditeurs académiques, qui plébiscitent ce standard pour les publications scientifiques~;
\item Nature collaborative facilitée par un éditeur de texte en ligne tel que \Marque{Overleaf}, offrant des fonctionnalités de correction et de commentaire\footnote{
\Marque{Overleaf} (\url{https://www.overleaf.com/}) est un éditeur \latexword{\LaTeX} qui combine un éditeur de code et un aperçu, tout en facilitant la collaboration en temps réel à l'aide de fonctionnalités de partage et de versionnement. La plateforme en ligne intègre des logiciels de gestion de références bibliographiques tels que \Marque{Zotero} et \Marque{Mendeley}.
}~;
\item Reconnaissance d'une typographie professionnelle grâce à une prise en charge complète des règles typographiques~;
\item Gestion facilitée de la bibliographie et des formules mathématiques~;
\item Personnalisation avancée à l'aide des commandes, enrichies par des modules complémentaires (\textit{packages})~;
\item Capacité à inclure du contenu et des annotations non visibles dans le rendu final~;
\item Aide et contribution au sein d'une communauté \latexword{\LaTeX} particulièrement active.
    \end{customitemize}%%Rédigé%%

    % Remerciements LaTeX
Dans une perspective plus personnelle, je tiens à exprimer ma profonde gratitude envers \textcolor{blue}{Jorge Mariano}\index{Mariano, Jorge|pagebf}, Ingénieur de recherche à l'Université Gustave Eiffel, pour sa contribution significative dans la conception d'un modèle \latexword{\LaTeX} conforme aux standards visuels de l'Université Gustave Eiffel, ainsi que pour son soutien régulier. Ce manuscrit intègre en partie ce modèle et a été agrémenté de modifications visant une personnalisation mieux adaptée aux spécificités de notre discipline et à nos préférences individuelles. L'adoption de \latexword{\LaTeX} a également été possible grâce aux conseils avisés de mon directeur de thèse, \textcolor{blue}{Alain L'Hostis}\index{L'Hostis, Alain|pagebf}, qui a su, dès le départ de mon parcours doctoral, me convaincre de l'intérêt de ce système et de ce langage informatique. Mes pensées vont à mon collègue et ami \textcolor{blue}{Iñigo Aguas Ardaiz}\index{Aguas Ardaiz, Iñigo|pagebf}, qui, en fin de recherche, m'a généreusement accordé de son temps précieux et apporté un secours inestimable dans la résolution du code. Dans un dernier temps, je tiens à remercier le soutien de l'Université Gustave Eiffel qui finance un accès professionnel à l'environnement de collaboration en \latexword{\LaTeX} \Marque{Overleaf}.%%Rédigé%%

    % Lien GitHub
Avec pour finalité de science ouverte et dès lors de partage des connaissances, de réutilisation et de collaboration, nous avons rendu le présent manuscrit, ainsi que le modèle personnalisé et le code utilisé, accessibles grâce à un dépôt \Marque{GitHub}\footnote{
    \Marque{GitHub} (\url{https://github.com/}) est une plateforme en ligne de gestion de développement de logiciels et un service d'hébergement des projets en libre accès.
}.%%Rédigé%%

    \bigskip
    \begin{tcolorbox}[colback=white!5!white,
                      colframe=blue!75!blue,
                      title=
                      \bigskip
                      \center{Dépôt \Marque{GitHub}}
                      \bigskip]
\center{\normalsize{\url{https://github.com/dylan-moinse/PhD_Thesis_Dylan_MOINSE_French}}}
    \end{tcolorbox}