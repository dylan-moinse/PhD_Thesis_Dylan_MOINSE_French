% Tableau T1
%%Rédigé%%
  \begin{table}[h!]
    \centering
    \renewcommand{\arraystretch}{1.5}
    \resizebox{\columnwidth}{!}{
    \begin{tabular}{p{0.1\columnwidth}p{0.5\columnwidth}p{0.4\columnwidth}}
      % \hline
      \rule{0pt}{15pt} \textcolor{blue}{\textbf{\small{ID}}} & \textcolor{blue}{\textbf{\small{Intitulé de la question}}} & \textcolor{blue}{\textbf{\small{Réponses}}}\\
      \hline
        \multicolumn{3}{l}{\textsl{\textbf{Thématique 1~: Ressenti global}} (\(T_{1}\))}\\
            \hdashline
    \small{\(Q_{14}\)} & \small{\textsl{Se déplacer à vélo dans votre commune est\dots}} & \small{\textsl{Désagréable / Agréable}}\\
        \hdashline
    \small{\(Q_{15}\)} & \small{\textsl{Le réseau d'itinéraires cyclables de ma commune me permet d'aller partout de façon rapide et directe}} & \small{\textsl{Pas du tout / Tout à fait}}\\
        \hdashline
    \small{\(Q_{16}\)} & \small{\textsl{Les conflits entre les personnes circulant à vélo et à pied sont\dots}} & \small{\textsl{Très fréquents / Très rares}}\\
        \hdashline
    \small{\(Q_{17}\)} & \small{\textsl{À vélo, les personnes conduisant des véhicules motorisés me respectent}} & \small{\textsl{Pas du tout / Tout à fait}}\\
        \hdashline
    \small{\(Q_{18}\)} & \small{\textsl{À vélo, je trouve que le trafic motorisé (volume et vitesse) est\dots}} & \small{\textsl{Insupportable / Pas du tout gênant}}\\
        \hdashline
    \small{\(Q_{19}\)} & \small{\textsl{Selon moi, dans ma commune, l'usage du vélo est\dots}} & \small{\textsl{Limité à certains / Très démocratisé}}\\
        \hline
    \end{tabular}}
    \caption*{}
    \vspace{5pt}
        \begin{flushright}\scriptsize
        Source~: \textcolor{blue}{\textcite{fub_barometre_2021}}
        \end{flushright}
        \end{table}

% Tableau T2
%%Rédigé%%
  \begin{table}[h!]
    \centering
    \renewcommand{\arraystretch}{1.5}
    \resizebox{\columnwidth}{!}{
    \begin{tabular}{p{0.1\columnwidth}p{0.5\columnwidth}p{0.4\columnwidth}}
      % \hline
      \rule{0pt}{15pt} \textcolor{blue}{\textbf{\small{ID}}} & \textcolor{blue}{\textbf{\small{Intitulé de la question}}} & \textcolor{blue}{\textbf{\small{Réponses}}}\\
      \hline
        \multicolumn{3}{l}{\textsl{\textbf{Thématique 2~: Sécurité}} (\(T_{2}\))}\\
            \hdashline
    \small{\(Q_{20}\)} & \small{\textsl{En général, quand je circule à vélo dans ma commune, je me sens\dots}} & \small{\textsl{En sécurité / En danger}}\\
        \hdashline
    \small{\(Q_{21}\)} & \small{\textsl{Je peux circuler à vélo en sécurité sur les grands axes dans ma commune}} & \small{\textsl{Pas du tout / Tout à fait}}\\
        \hdashline
    \small{\(Q_{22}\)} & \small{\textsl{Je peux circuler à vélo en sécurité dans les rues résidentielles}} & \small{\textsl{Pas du tout / Tout à fait}}\\
        \hdashline
    \small{\(Q_{23}\)} & \small{\textsl{Je peux rejoindre à vélo en sécurité les communes voisines}} & \small{\textsl{Pas du tout / Tout à fait}}\\
        \hdashline
    \small{\(Q_{24}\)} & \small{\textsl{Selon moi, traverser un carrefour ou un rond-point est\dots}} & \small{\textsl{Toujours dangereux / Jamais dangereux}}\\
        \hdashline
    \small{\(Q_{25}\)} & \small{\textsl{Pour les enfants et les personnes âgées, circuler à vélo est\dots}} & \small{\textsl{Très dangereux / Très sûr}}\\
        \hline
    \end{tabular}}
    \caption*{}
    \vspace{5pt}
        \begin{flushright}\scriptsize
        Source~: \textcolor{blue}{\textcite{fub_barometre_2021}}
        \end{flushright}
        \end{table}

% Tableau T3
%%Rédigé%%
  \begin{table}[h!]
    \centering
    \renewcommand{\arraystretch}{1.5}
    \resizebox{\columnwidth}{!}{
    \begin{tabular}{p{0.1\columnwidth}p{0.5\columnwidth}p{0.4\columnwidth}}
      % \hline
      \rule{0pt}{15pt} \textcolor{blue}{\textbf{\small{ID}}} & \textcolor{blue}{\textbf{\small{Intitulé de la question}}} & \textcolor{blue}{\textbf{\small{Réponses}}}\\
      \hline
        \multicolumn{3}{l}{\textsl{\textbf{Thématique 3~: Confort}} (\(T_{3}\))}\\
            \hdashline
    \small{\(Q_{26}\)} & \small{\textsl{Selon moi, les itinéraires cyclables sont\dots}} & \small{\textsl{Pas du tout confortables / Très confortables}}\\
        \hdashline
    \small{\(Q_{27}\)} & \small{\textsl{L'entretien des itinéraires cyclables est\dots}} & \small{\textsl{Très mauvais / Très bon}}\\
        \hdashline
    \small{\(Q_{28}\)} & \small{\textsl{Les directions à vélo sont correctement indiquées par des panneaux}} & \small{\textsl{Pas du tout / Tout à fait}}\\
        \hdashline
    \small{\(Q_{29}\)} & \small{\textsl{Lors de travaux sur les itinéraires cyclables, une solution alternative sûre est proposée}} & \small{\textsl{Jamais / Toujours}}\\
        \hdashline
    \small{\(Q_{30}\)} & \small{\textsl{À vélo, je suis autorisé à emprunter les voies à sens unique à contre-sens}} & \small{\textsl{Jamais / Toujours}}\\
        \hline
    \end{tabular}}
    \caption*{}
    \vspace{5pt}
        \begin{flushright}\scriptsize
        Source~: \textcolor{blue}{\textcite{fub_barometre_2021}}
        \end{flushright}
        \end{table}

% Tableau T4
%%Rédigé%%
  \begin{table}[h!]
    \centering
    \renewcommand{\arraystretch}{1.5}
    \resizebox{\columnwidth}{!}{
    \begin{tabular}{p{0.1\columnwidth}p{0.5\columnwidth}p{0.4\columnwidth}}
      % \hline
      \rule{0pt}{15pt} \textcolor{blue}{\textbf{\small{ID}}} & \textcolor{blue}{\textbf{\small{Intitulé de la question}}} & \textcolor{blue}{\textbf{\small{Réponses}}}\\
      \hline
        \multicolumn{3}{l}{\textsl{\textbf{Thématique 4~: Efforts de la commune}} (\(T_{4}\))}\\
            \hdashline
    \small{\(Q_{31}\)} & \small{\textsl{Selon moi, les efforts faits en faveur du vélo par la ville sont\dots}} & \small{\textsl{Inexistants / Importants}}\\
        \hdashline
    \small{\(Q_{32}\)} & \small{\textsl{La communication en faveur des déplacements à vélo est\dots}} & \small{\textsl{Inexistante / Importante}}\\
        \hdashline
    \small{\(Q_{33}\)} & \small{\textsl{La mairie est à l'écoute des besoins des usagers du vélo, elle les implique dans ses réflexions sur les mobilités et les projets d'aménagement}} & \small{\textsl{Jamais / Toujours}}\\
        \hdashline
    \small{\(Q_{34}\)} & \small{\textsl{Selon moi, le stationnement des véhicules motorisés (voitures, camions, motos\dots) sur les itinéraires cyclables est\dots}} & \small{\textsl{Très fréquent / Très rares}}\\
        \hline
    \end{tabular}}
    \caption*{}
    \vspace{5pt}
        \begin{flushright}\scriptsize
        Source~: \textcolor{blue}{\textcite{fub_barometre_2021}}
        \end{flushright}
        \end{table}

% Tableau T5
%%Rédigé%%
  \begin{table}[h!]
    \centering
    \renewcommand{\arraystretch}{1.5}
    \resizebox{\columnwidth}{!}{
    \begin{tabular}{p{0.1\columnwidth}p{0.5\columnwidth}p{0.4\columnwidth}}
      % \hline
      \rule{0pt}{15pt} \textcolor{blue}{\textbf{\small{ID}}} & \textcolor{blue}{\textbf{\small{Intitulé de la question}}} & \textcolor{blue}{\textbf{\small{Réponses}}}\\
      \hline
        \multicolumn{3}{l}{\textsl{\textbf{Thématique 5~: Services et stationnements}} (\(T_{5}\))}\\ 
            \hdashline
    \small{\(Q_{35}\)} & \small{\textsl{Dans la commune ou à proximité, trouver un stationnement vélo adapté à mon besoin (durée, sécurité\dots) est\dots}} & \small{\textsl{Impossible / Très facile}}\\
        \hdashline
    \small{\(Q_{36}\)} & \small{\textsl{Stationner son vélo en gare ou à une station de transports en commun est\dots}} & \small{\textsl{Impossible / Très facile}}\\
        \hdashline
    \small{\(Q_{37}\)} & \small{\textsl{Louer un vélo pour quelques heures ou pour plusieurs mois est\dots}} & \small{\textsl{Impossible / Très facile}}\\
        \hdashline
    \small{\(Q_{38}\)} & \small{\textsl{Dans la commune ou à proximité, trouver un magasin ou un atelier de réparation est\dots}} & \small{\textsl{Impossible / Très facile}}\\
        \hdashline
    \small{\(Q_{39}\)} & \small{\textsl{Selon moi, les vols de vélos sont\dots}} & \small{\textsl{Très fréquents / Très rares}}\\
      \hline
    \end{tabular}}
    \caption*{}
    \vspace{5pt}
        \begin{flushleft}\scriptsize
        \textcolor{blue}{Lecture~:} dans le \textsl{Baromètre des Villes Cyclables}, 26 questions (de \(Q_{14}\) à \(Q_{39}\)) servent à déterminer la notation moyenne de cyclabilité des communes, au travers de 5 thématiques (de \(T_{1}\) à \(T_{5}\)).
        \end{flushleft}
        \begin{flushright}\scriptsize
        Source~: \textcolor{blue}{\textcite{fub_barometre_2021}}
        \end{flushright}
        \end{table}