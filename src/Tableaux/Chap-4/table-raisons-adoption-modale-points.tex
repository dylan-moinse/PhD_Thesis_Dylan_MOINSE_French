% Tableau Raisons adoption par points
%%Rédigé%%
    \begin{table}[h!]
    \centering
    \renewcommand{\arraystretch}{1.5}
    \resizebox{\columnwidth}{!}{
    \begin{tabular}{p{0.07\columnwidth}p{0.06\columnwidth}p{0.4\columnwidth}p{0.1\columnwidth}p{0.1\columnwidth}p{0.1\columnwidth}}
        %\hline
    \rule{0pt}{15pt} \small{\textcolor{blue}{\textbf{Rang}}} & \small{\textcolor{blue}{\textbf{ID}}} & \small{\textcolor{blue}{\textbf{Raisons classées}}} & \small{\textcolor{blue}{\textbf{Points}}} & \small{\textcolor{blue}{\textbf{Top~3}}} & \small{\textcolor{blue}{\textbf{Effectif}}}\\
        \hline
\small{1} & \small{\(F\)} & \small{Sensibilité environnementale} & \small{\textbf{492}} & \small{105} & \small{137}\\
\small{2} & \small{\(E\)} & \small{Distances trop longues à pied} & \small{\textbf{487}} & \small{107} & \small{128}\\
\small{3} & \small{\(G\)} & \small{Gains de flexibilité} & \small{\textbf{405}} & \small{90} & \small{108}\\
        \hdashline
\small{4} & \small{\(H\)} & \small{Sensation de liberté} & \small{\textbf{266}} & \small{57} & \small{83}\\
\small{5} & \small{\(M\)} & \small{Prendre l'air} & \small{\textbf{255}} & \small{54} & \small{87}\\
\small{6} & \small{\(C\)} & \small{Coûts économiques de l'automobile} & \small{\textbf{179}} & \small{39} & \small{56}\\
\small{7} & \small{\(O\)} & \small{Autres raisons déclarées} & \small{\textbf{164}} & \small{33} & \small{51}\\
\small{8} & \small{\(L\)} & \small{Itinéraire porte-à-porte} & \small{\textbf{152}} & \small{33} & \small{49}\\
\small{9} & \small{\(B\)} & \small{Confort du déplacement} & \small{\textbf{142}} & \small{31} & \small{46}\\
\small{10} & \small{\(N\)} & \small{Présence d'un réseau cyclable} & \small{\textbf{126}} & \small{29} & \small{43}\\
\small{11} & \small{\(K\)} & \small{Absence d'alternative} & \small{\textbf{116}} & \small{25} & \small{30}\\
\small{12} & \small{\(I\)} & \small{Ludique} & \small{\textbf{112}} & \small{18} & \small{50}\\
\small{13} & \small{\(D\)} & \small{Curiosité} & \small{\textbf{37}} & \small{5} & \small{16}\\
\small{14} & \small{\(J\)} & \small{Mimétisme} & \small{\textbf{14}} & \small{3} & \small{6}\\
\small{15} & \small{\(A\)} & \small{Bouche-à-oreille} & \small{\textbf{2}} & \small{0} & \small{2}\\
    \hdashline
\multicolumn{3}{l}{\small{\textbf{Moyenne générale}}} & \small{\textbf{197}} & \small{\textbf{59}} & \small{\textbf{-}}\\
        \hline
        \end{tabular}}
    \caption{Répartition par point des raisons de l'adoption intermodale de la mobilité individuelle légère.}
    \label{table-chap4:raisons-adoption-modale-points}
        \vspace{5pt}
        \begin{flushleft}\scriptsize{
        \textcolor{blue}{Note~:} la colonne \textsl{Points} fait référence au nombre de points attribué de manière dégressive à chaque raison, la première position du classement octroyant cinq points, contre un point pour la cinquième position. La colonne \textsl{Top~3} correspond au nombre de fois où l'option a été placée dans les trois premiers choix parmi les réponses apportées.
        \\
        \textcolor{blue}{Lecture~:} à partir des 217 réponses analysées, les résultats montrent que la sensibilité environnement la perception des distances trop longues à pied sont les deux principales raisons qui motivent l'adoption intermodale de la mobilité individuelle légère, suivies par les gains de flexibilité qui complètent le trio de tête des motivations exprimées.
        }\end{flushleft}
        \begin{flushright}\scriptsize{
        Auteur~: \textcolor{blue}{Dylan Moinse (2024)}
        }\end{flushright}
        \end{table}