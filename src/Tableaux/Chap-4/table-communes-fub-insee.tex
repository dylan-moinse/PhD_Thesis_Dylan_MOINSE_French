% Tableau Liste communes
%%Rédigé%%
    \begin{table}[h!]
    \centering
    \renewcommand{\arraystretch}{1.5}
    \resizebox{\columnwidth}{!}{
    \begin{tabular}{p{0.34\columnwidth}p{0.56\columnwidth}p{0.1\columnwidth}}
        %\hline
    \rule{0pt}{15pt} \small{\textcolor{blue}{{\textbf{Région}}}} & \small{\textcolor{blue}{{\textbf{Villes centres}}}} & \small{\textcolor{blue}{{\textbf{Effectif}}}}\\
        \hline
\multirow{1.5}{*}{\small{Auvergne-Rhône-Alpes}} & \small{Annecy, Chambéry, Clermont-Ferrand, Grenoble, Lyon, Saint-Étienne et Valence} & \multirow{1.5}{*}{\small{7}}\\
\small{Bourgogne-Franche-Comté} & \small{Besançon et Dijon} & \small{2}\\
\small{Bretagne} & \small{Brest, Lorient, Rennes et Saint-Nazaire} & \small{4}\\
\small{Centre-Val de Loire} & \small{Orléans et Tours} & \small{2}\\
\multirow{1.5}{*}{\small{Grand Est}} & \small{Metz, Mulhouse, Nancy, Reims, Strasbourg et Troyes} & \multirow{1.5}{*}{\small{6}}\\
\multirow{1.5}{*}{\small{Hauts-de-France}} & \small{Amiens, Arras, Beauvais, Douai, Dunkerque, Lille et Valenciennes} & \multirow{1.5}{*}{\small{7}}\\
\small{Île-de-France} & \small{Paris} & \small{1}\\
\small{Normandie} & \small{Caen, Le Havre et Rouen} & \small{3}\\
\multirow{1.5}{*}{\small{Nouvelle-Aquitaine}} & \small{Angoulême, Bayonne, Bordeaux, La Rochelle, Limoges et Poitiers} & \multirow{1.5}{*}{\small{6}}\\
\small{Occitanie} & \small{Montpellier, Nîmes, Perpignan et Toulouse} & \small{4}\\
\small{Pays de la Loire} & \small{Angers, Le Mans et Nantes} & \small{3}\\
\multirow{1.5}{*}{\small{Provence-Alpes-Côte d'Azur}} & \small{Aix-en-Provence, Avignon, Marseille, Nice et Toulon} & \multirow{1.5}{*}{\small{5}}\\
\small{La Réunion} & \small{Saint-Denis, Saint-Paul et Saint-Pierre} & \small{3}\\
        \hline
        \end{tabular}}
    \caption{Liste des 53 communes françaises examinées dans le cadre de l'étude sur la cyclabilité mise en relation avec la pratique genrée du vélo et de la micro-mobilité.}
    \label{table-chap4:communes-fub-insee}
        \vspace{5pt}
        \begin{flushleft}\scriptsize
        \textcolor{blue}{Lecture~:} ce tableau recense les 53 communes françaises réparties dans 13 régions, incluant des métropoles et des villes moyennes, afin d'étudier les liens entre leur cyclabilité et la pratique genrée du vélo.
        \end{flushleft}
        \begin{flushright}\scriptsize{
        Jeux de données~: \textsl{MOBPro} \textcolor{blue}{\autocite{insee_documentation_2023}} et \textsl{Baromètre des Villes Cyclables} \textcolor{blue}{\autocite{fub_barometre_2021}}
        \\
        Auteur~: \textcolor{blue}{Dylan Moinse (2023)}
        }\end{flushright}
        \end{table}