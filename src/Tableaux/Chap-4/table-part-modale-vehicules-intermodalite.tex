% Tableau Part modale intermodale - modes de transfert
%%Rédigé%%
    \begin{table}[h!]
    \centering
    \renewcommand{\arraystretch}{1.5}
    \resizebox{\columnwidth}{!}{
    \begin{tabular}{p{0.50\columnwidth}p{0.15\columnwidth}p{0.17\columnwidth}p{0.18\columnwidth}}
        %\hline
    \rule{0pt}{15pt} \small{\textbf{\textcolor{blue}{Type de véhicule}}} & \small{\textbf{\textcolor{blue}{Observation}}} & \small{\textbf{\textcolor{blue}{Questionnaire}}} & \small{\textbf{\textcolor{blue}{Embarquement}}}\\
        \hline
\small{\textbf{Tous types de mobilité individuelle légère confondus}} & \multirow{1.5}{*}{\small{\textbf{100,00~\%}}} & \multirow{1.5}{*}{\small{\textbf{100,00~\%}}} & \multirow{1.5}{*}{\small{\textbf{77,88~\%}}}\\
        \hdashline
\small{\textbf{Tous types de vélo personnel}} & \small{\textbf{43,67~\%}} & \small{\textbf{67,74~\%}} & \small{\textbf{78,91~\%}}\\
\small{Vélo conventionnel} & \small{31,79~\%} & \small{49,77~\%} & \small{75,93~\%}\\
\small{Vélo pliant} & \small{11,88~\%} & \small{11,06~\%} & \small{100,00~\%}\\
\small{Vélo à assistance électrique (\acrshort{VAE})} & \small{-} & \small{3,69~\%} & \small{50,00~\%}\\
\small{Vélo pliant électrique} & \small{-} & \small{2,76~\%} & \small{100,00~\%}\\
\small{Vélo cargo} & \small{-} & \small{0,46~\%} & \small{0,00~\%}\\
        \hdashline
\small{\textbf{Tous types de trottinette personnelle}} & \small{\textbf{54,49~\%}} & \small{\textbf{22,12~\%}} & \small{\textbf{100,00~\%}}\\
\small{Trottinette électrique personnelle (\acrshort{TEP})} & \small{44,44~\%} & \small{17,97~\%} & \small{100,00~\%}\\
\small{Trottinette mécanique} & \small{10,05~\%} & \small{4,15~\%} & \small{100,00~\%}\\
        \hdashline
\small{\textbf{Autres types d'\acrfull{EDP}}} & \multirow{1.5}{*}{\small{\textbf{1,84~\%}}} & \multirow{1.5}{*}{\small{\textbf{2,76~\%}}} & \multirow{1.5}{*}{\small{\textbf{83,33~\%}}}\\
\small{\textsl{Skateboard}} & \small{1,06~\%} & \small{1,38~\%} & \small{100,00~\%}\\
\small{Monoroue} & \small{0,77~\%} & \small{0,92~\%} & \small{100,00~\%}\\
\small{Gyropode} & \small{-} & \small{0,46~\%} & \small{0,00~\%}\\
        \hdashline
\small{\textbf{Tous types de véhicule partagé}} & \small{\textbf{-}} & \small{\textbf{7,37~\%}} & \small{\textbf{0,00~\%}}\\
\small{Vélo en libre-service (\acrshort{VLS})} & \small{-} & \small{6,45~\%} & \small{0,00~\%}\\
\small{Vélo électrique en \textsl{free-floating} (\acrshort{VFF})} & \small{-} & \small{0,46~\%} & \small{0,00~\%}\\
\small{Trottinette électrique en \textsl{free-floating} (\acrshort{TEFF})} & \multirow{1.5}{*}{\small{-}} & \multirow{1.5}{*}{\small{0,46~\%}} & \multirow{1.5}{*}{\small{0,00~\%}}\\
        \hline
        \end{tabular}}
    \caption{Part de chaque véhicule de transfert composant la mobilité individuelle légère, en France.}
    \label{table-chap4:part-modale-vehicules-intermodalite}
        \vspace{5pt}
        \begin{flushleft}\scriptsize{
        \textcolor{blue}{Note~:} la colonne \textsl{Observation} se réfère au sous-échantillon obtenu à partir de l'observation quantitative des voyageur·se·s (1~035 comptages), tandis que la colonne \textsl{Questionnaire} se rapporte aux déplacements intermodaux déclarés par les participant·e·s (217 réponses) et la modalité \textsl{Embarquement}, liée au questionnaire, quantifie la part de véhicules emportés à bord du transport public.
        \\
        \textcolor{blue}{Lecture~:} la répartition modale des cycles utilisés pour les transferts intermodaux montre que la trottinette électrique et le vélo, notamment pliant, à usage personnel, ont les parts les plus importantes. Le véhicule électrique et le vélo pliant, conventionnel comme électrique, ainsi que les dispositifs de mobilité légère tels que le \textsl{skateboard} et le monoroue, affichent un taux d'embarquement à 100~\%.
        }\end{flushleft}
        \begin{flushright}\scriptsize
        Auteur~: \textcolor{blue}{Dylan Moinse (2022)}
        \end{flushright}
        \end{table}