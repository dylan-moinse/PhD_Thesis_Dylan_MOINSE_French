% Tableau Typologie des gares HdF
%%Rédigé%%
    \begin{table}[h!]
    \centering
    \renewcommand{\arraystretch}{1.5}
    \resizebox{\columnwidth}{!}{
    \begin{tabular}{p{0.27\columnwidth}p{0.13\columnwidth}p{0.60\columnwidth}}
        %\hline
    \rule{0pt}{15pt} \small{\textbf{\textcolor{blue}{Gare ou halte}}} & \small{\textbf{\textcolor{blue}{\acrshort{DRG}}}} & \small{\textbf{\textcolor{blue}{Contextualisation de la classe}}}\\
        \hline
\small{Creil} & \multirow{2}{*}{\small{Profil~\(a\)}} & \multirow{2}{*}{\small{\textsl{Gares de pôles régionaux}}}\\
    \small{Lille Flandres} & & \\
        \hdashline
\small{Armentières} & \multirow{3}{*}{\small{Profil~\(b\)}} & \multirow{3}{*}{\small{\textsl{Gares de pôles intermédiaires}}}\\
    \small{Béthune} & & \\
    \small{Dunkerque} & & \\
        \hdashline
\small{Le Poirier Université} & \multirow{4}{*}{\small{Profil~\(c\)}} & \multirow{4}{*}{\small{\textsl{Gares de rabattement vers les centralités urbaines}}}\\
\small{Lesquin} & & \\
\small{Lille CHR} & & \\
\small{Vis-à-Marles} & & \\
        \hline
        \end{tabular}}
    \caption{Réutilisation du référentiel \Guillemets{Segment DRG} appliqué aux neuf gares explorées dans la région Hauts-de-France.}
    \label{table-chap4:typologie-gares-hdf}
        \vspace{5pt}
        \begin{flushleft}\scriptsize{
        \textcolor{blue}{Lecture~:} ce tableau applique la typologie des segments aux neuf gares des Hauts-de-France explorées dans notre investigation, en les classant en trois profils distincts~: les profils~\(a\), \(b\) et \(c\).
        }\end{flushleft}
        \begin{flushright}\scriptsize
        Jeux de données~: \textcolor{blue}{\textcite{sncf_gares__connexions_gares_2024}}\index{SNCF Gares \& Connexions@\textsl{SNCF Gares \& Connexions}|pagebf}
        \\
        Auteur~: \textcolor{blue}{Dylan Moinse (2022)}
        \end{flushright}
        \end{table}