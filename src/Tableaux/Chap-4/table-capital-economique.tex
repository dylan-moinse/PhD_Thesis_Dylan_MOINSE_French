% Tableau Capital économique PCS
%%Rédigé%%
    \begin{table}[h!]
    \centering
    \renewcommand{\arraystretch}{1.5}
    \resizebox{\columnwidth}{!}{
    \begin{tabular}{p{0.61\columnwidth}p{0.15\columnwidth}p{0.12\columnwidth}p{0.12\columnwidth}}
        %\hline
    \rule{0pt}{15pt} \small{\textcolor{blue}{\textbf{Situation professionnelle}}} & \small{\textcolor{blue}{\textbf{Enquête}}} & \small{\textcolor{blue}{\textbf{Train}}} & \small{\textcolor{blue}{\textbf{France}}}\\
        \hline
    \multicolumn{4}{l}{\textbf{\textcolor{blue}{\small{Statuts}}}}\\
\small{Actif·ve occupé·e à temps plein} & \textbf{\small{67,74~\%}}& \multirow{2}{*}{\small{47,80~\%}} & \small{41,30~\%}\\
\small{Actif·ve occupé·e à temps partiel} & \textbf{\small{8,29~\%}}& & \small{8,70~\%}\\
\small{Actif·ve inoccupé·e} & \textbf{\small{3,23~\%}}& \small{4,40~\%} & \small{7,30~\%}\\
\small{Étudiant·e en formation} & \textbf{\small{3,69~\%}}& \multirow{2}{*}{\small{23,00~\%}} & \small{6,30~\%}\\
\small{Étudiant·e salarié·e} & \textbf{\small{14,29~\%}}& & \small{2,70~\%}\\
\small{Retraité·e} & \textbf{\small{2,76~\%}}& \small{11,00~\%} & \small{20,00~\%}\\
\small{Autre inactif·ve} & \textbf{\small{0,00~\%}} & \small{1,70~\%} & \small{6,00~\%}\\
        \hdashline
    \multicolumn{4}{l}{\textbf{\textcolor{blue}{\small{\acrfull{PCS}}}}}\\
\small{Agriculteur·rice·s exploitant·e·s} & \textbf{\small{0,00~\%}} & \small{0,93~\%} & \small{1,60~\%}\\
\small{Artisan·e·s, commerçant·e·s et chef·fe·s d'entreprise} & \textbf{\small{3,64~\%}} & \small{6,96~\%} & \small{6,80~\%}\\
\small{Cadres et professions intellectuelles supérieures} & \textbf{\small{66,06~\%}} & \small{41,07~\%} & \small{21,70~\%}\\
\small{Professions intermédiaires} & \textbf{\small{9,09~\%}}& \small{7,66~\%} & \small{24,60~\%}\\
\small{Employé·e·s} & \textbf{\small{17,58~\%}}& \multirow{2}{*}{\small{43,39~\%}} & \small{26,00~\%}\\
\small{Ouvrier·ère·s} & \textbf{\small{3,64~\%}} & & \small{18,90~\%}\\
        \hdashline
    \multicolumn{4}{l}{\textbf{\textcolor{blue}{\small{Revenus mensuels disponibles bruts}}}}\\
\small{Moyenne} & \textbf{\small{3~058~\euro}} & \small{-} & \small{3~250~\euro}\\
\small{Médiane} & \textbf{\small{2~850~\euro}} & \small{-} & \small{2~250~\euro}\\
        \hline
        \end{tabular}}
    \caption{Une surreprésentation de navetteur·se·s intermodaux·les détenteur·rice·s de capitaux économiques, culturels et symboliques élevés.}
    \label{table-chap4:capital-economique}
        \vspace{5pt}
        \begin{flushleft}\scriptsize
        \textcolor{blue}{Lecture~:} ce tableau montre une surreprésentation des actif·ve·s à temps plein, des cadres, et des revenus élevés parmi les usager·ère·s intermodaux·les, par rapport à la moyenne nationale.
        \end{flushleft}
        \begin{flushright}\scriptsize{
        Jeux de données~: \textcolor{blue}{\textcite{sncf_repartition_2017}}\index{SNCF@\textsl{SNCF}|pagebf}, \textcolor{blue}{\textcite{insee_categorie_2024}}\index{Insee@\textsl{Insee}|pagebf}, \textcolor{blue}{\textcite{insee_evolution_2023}}\index{Insee@\textsl{Insee}|pagebf} et \textcolor{blue}{\textcite{insee_niveau_2024}}\index{Insee@\textsl{Insee}|pagebf}
        \\
        Auteur~: \textcolor{blue}{Dylan Moinse (2024)}
        }\end{flushright}
        \end{table}