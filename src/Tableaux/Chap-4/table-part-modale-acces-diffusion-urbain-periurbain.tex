% Tableau Part modale intermodale - modes de transfert
%%Rédigé%%
    \begin{table}[h!]
    \centering
    \renewcommand{\arraystretch}{1.5}
    \resizebox{\columnwidth}{!}{
    \begin{tabular}{p{0.4\columnwidth}p{0.18\columnwidth}p{0.14\columnwidth}p{0.14\columnwidth}p{0.14\columnwidth}}
        %\hline
    \rule{0pt}{15pt} \small{\textbf{\textcolor{blue}{Type de véhicule}}} & \small{\textbf{\textcolor{blue}{Rabattement}}} & \small{\textbf{\textcolor{blue}{Couronne}}} & \small{\textbf{\textcolor{blue}{Diffusion}}} & \small{\textbf{\textcolor{blue}{Couronne}}}\\
        \hline
\small{\textbf{Tous types de mobilité individuelle légère confondus}} & \multirow{1.5}{*}{\small{\textbf{78,34~\%}}} & \multirow{1.5}{*}{\small{\textbf{31,86~\%}}} & \multirow{1.5}{*}{\small{\textbf{78,80~\%}}} & \multirow{1.5}{*}{\small{\textbf{20,59~\%}}}\\
        \hdashline
\small{Vélo conventionnel} & \small{80,51~\%} & \small{34,58~\%} & \small{65,25~\%} & \small{20,56~\%}\\
\small{Vélo électrique (\acrshort{VAE})} & \small{76,92~\%} & \small{60,00~\%} & \small{61,54~\%} & \small{20,00~\%}\\
\small{Vélo pliant} & \small{80,95~\%} & \small{21,74~\%} & \small{100,00~\%} & \small{26,09~\%}\\
\small{Trottinette électrique (\acrshort{TEP})} & \small{80,43~\%} & \small{38,46~\%} & \small{100,00~\%} & \small{25,64~\%}\\
\small{Trottinette mécanique} & \small{77,78~\%} & \small{20,00~\%} & \small{100,00~\%} & \small{20,00~\%}\\
\small{Mobilité partagée} & \small{40,00~\%} & \small{0,00~\%} & \small{70,00~\%} & \small{0,00~\%}\\
        \hdashline
\small{Marche combinée} & \small{36,17~\%} & \small{17,24~\%} & \small{63,83~\%} & \small{10,34~\%}\\
        \hline
        \end{tabular}}
    \caption{Usage intermodal de la mobilité individuelle légère en rabattement depuis et en diffusion vers les territoires périurbains.}
    \label{table-chap4:part-modale-acces-diffusion-urbain-periurbain}
        \vspace{5pt}
        \begin{flushleft}\scriptsize{
        \textcolor{blue}{Lecture~:} parmi les utilisateur·rice·s combinant la mobilité individuelle légère avec le système de transport en commun et ayant déclaré leur déplacement dans le cadre du questionnaire, 78~\% d'entre elleux circulent avec un véhicule léger en pré-acheminement et 32~\% ont pour origine un territoire périurbain, tandis que 79~\% y ont recours en post-acheminement et 21~\% ont pour destination un territoire périurbain.
        }\end{flushleft}
        \begin{flushright}\scriptsize
        Auteur~: \textcolor{blue}{Dylan Moinse (2023)}
        \end{flushright}
        \end{table}