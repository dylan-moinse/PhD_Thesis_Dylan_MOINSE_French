% Tableau Typologie pauses
%%Rédigé%%
        \begin{table}[h!]
        \centering
        \renewcommand{\arraystretch}{1.5}
        \resizebox{\columnwidth}{!}{
        \begin{tabular}{p{0.58\columnwidth}p{0.21\columnwidth}p{0.21\columnwidth}}
        %\hline
    \rule{0pt}{15pt} \textcolor{blue}{\textbf{Typologie des pauses}} & \textcolor{blue}{\textbf{Effectif}} & \textcolor{blue}{\textbf{Part}}\\
        \hline
\small{Achats} & \small{82} & \small{74,5~\%} \\
\small{Rencontres sociales et accompagnement} & \small{30} & \small{27,3~\%} \\
\small{Activités professionnelles} & \small{27} & \small{24,5~\%} \\
\small{Loisirs} & \small{23} & \small{20,9~\%} \\
\small{Tâches administratives} & \small{17} & \small{15,5~\%} \\
\small{Visites et promenades} & \small{16} & \small{14,5~\%} \\
        \hline
        \end{tabular}}
    \caption{Raisons avancées pour les 110 pauses déclarées, lors des déplacements intermodaux, par les cyclo-voyageur·se·s.}
    \label{table-chap5:motifs-pauses}
        \vspace{5pt}
        \begin{flushleft}\scriptsize{
        \textcolor{blue}{Note~:} la question concernée étant à choix multiples, la part totale est supérieure à 100~\%.
        \\
        \textcolor{blue}{Lecture~:} les achats constituent la principale raison des pauses lors des déplacements intermodaux, suivis des rencontres sociales et de l'accompagnement, et des activités professionnelles. Ce tableau souligne ainsi la diversité des motifs associés aux interruptions de parcours.
        }\end{flushleft}
        \begin{flushright}\scriptsize{
        Auteur~: \textcolor{blue}{Dylan Moinse (2023)}
        }\end{flushright}
        \end{table}