% Tableau accessibilité aux POIs
%%Rédigé%%
    \begin{table}[h!]
    \centering
    \renewcommand{\arraystretch}{1.5}
    \resizebox{\columnwidth}{!}{
    \begin{tabular}{p{0.35\columnwidth}p{0.12\columnwidth}p{0.17\columnwidth}p{0.12\columnwidth}p{0.12\columnwidth}p{0.12\columnwidth}}
        %\hline
    \rule{0pt}{15pt} \small{\textbf{\textcolor{blue}{Équipements et services}}} & \small{\textbf{\textcolor{blue}{\textless1 km.}}} & \small{\textbf{\textcolor{blue}{{[}1~;~4{[} km.}}} & \small{\textbf{\textcolor{blue}{\textless4 km.}}} & \small{\textbf{\textcolor{blue}{\geq 4 km.}}} & \small{\textbf{\textcolor{blue}{Région}}}\\
        \hline
    \multicolumn{6}{l}{\textbf{Tous types de \acrshort{POIs}}}\\
\small{Effectif} & \small{31~743} & \small{36~689} & \small{68~432} & \small{36~009} & \small{104~441}\\
\small{Part} & \small{30,39~\%} & \small{35,13~\%} & \small{65,52~\%} & \small{34,48~\%} & \small{100~\%}\\
\small{Densité (km\textsuperscript{2})} & \small{63,23} & \small{23,21} & \small{32,87} & \small{9,96} & \small{18,35}\\
        \hdashline
    \multicolumn{6}{l}{\textbf{\acrshort{POIs} de \Guillemets{proximité}}}\\
\small{Effectif} & \small{18~976} & \small{23~171} & \small{42~147} & \small{26~371} & \small{68~518}\\
\small{Part} & \small{27,69~\%} & \small{33,82~\%} & \small{61,51~\%} & \small{38,49~\%} & \small{100~\%}\\
\small{Densité (km\textsuperscript{2})} & \small{37,80} & \small{14,66} & \small{20,24} & \small{7,29} & \small{12,04}\\
        \hdashline
    \multicolumn{6}{l}{\textbf{\acrshort{POIs} \Guillemets{intermédiaires}}}\\
\small{Effectif} & \small{9~392} & \small{9~651} & \small{19~043} & \small{7~716} & \small{26~759}\\
\small{Part} & \small{35,10~\%} & \small{36,07~\%} & \small{71,16~\%} & \small{28,84~\%} & \small{100~\%}\\
\small{Densité (km\textsuperscript{2})} & \small{18,71} & \small{6,11} & \small{9,15} & \small{2,13} & \small{4,70}\\
        \hdashline
    \multicolumn{6}{l}{\textbf{\acrshort{POIs} \Guillemets{supérieurs}}}\\
\small{Effectif} & \small{3~375} & \small{3~867} & \small{7~242} & \small{1~922} & \small{9~164}\\
\small{Part} & \small{36,83~\%} & \small{42,20~\%} & \small{79,03~\%} & \small{20,97~\%} & \small{100~\%}\\
\small{Densité (km\textsuperscript{2})} & \small{6,72} & \small{2,45} & \small{3,48} &	\small{0,53} & \small{1,61}\\
        \hline
        \end{tabular}}
    \caption{Accessibilité aux points d'intérêt centrée sur les quartiers de gare piétonniers et cyclables.}
    \label{table-chap5:accessibilite-poi}
        \vspace{5pt}
        \begin{flushleft}\scriptsize{
        \textcolor{blue}{Note~:} les quartiers de gare accessibles à vélo et en micro-mobilité s'étendent jusqu'à trois kilomètres, hormis les aires d'influence des six pôles d'échange multimodaux, dont le rayon mesure quatre kilomètres.
        \\
        \textcolor{blue}{Lecture~:} les points d'intérêt de \Guillemets{proximité} et \Guillemets{intermédiaires} sont majoritairement accessibles dans des rayons limités, tandis que les équipements de plus grande envergue sont davantage concentrés à des distances accessibles en cycle. Cela met en évidence une hiérarchie des équipements en fonction de leur distance d'accès.
        }\end{flushleft}
        \begin{flushright}\scriptsize
        Jeux de données~: \acrfull{BPE} issue de l'\textcolor{blue}{\textcite{insee_base_2021}}\index{Insee@\textsl{Insee}|pagebf}
        \\
        Auteur~: \textcolor{blue}{Dylan Moinse (2024)}
        \end{flushright}
        \end{table}