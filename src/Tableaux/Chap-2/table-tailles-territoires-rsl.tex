% Tableau types de territoires analysés RSL
%%Rédigé%%
        \begin{table}[h!]
    \centering
    \renewcommand{\arraystretch}{1.5}
    \resizebox{\columnwidth}{!}{
    \begin{tabular}{p{1\columnwidth}}
        %\hline
    \rule{0pt}{15pt} \small{\textbf{\textcolor{blue}{Agglomérations et communes}}}\\
        \hline
    \small{\textbf{\textcolor{blue}{Plus de 10~000~000 d'habitant·e·s (88 références)}}}\\
\small{Beijing (21), Nanjing (19), Shenzhen (9), Shanghai (9), Séoul (6), New York City (4), Chengdu (3), Los Angeles (3), New Delhi (3), Xi'an (3), Chicago (2), Île-de-France (2), Mumbai (2), Bogotá (1), Johannesbourg-Pretoria (1), Osaka (1), Manille (1), Rio de Janeiro (1)}\\
        \hdashline
    \small{\textbf{\textcolor{blue}{Entre 3~000~000 et 10~000~000 d'habitant·e·s (53 références)}}}\\
\small{Washington D.C. (7), Berlin (4), Boston (4), Taipei (4), San Francisco (3), Seattle (3), Suzhou (3), Kaohsiung (2), Melbourne (2), Minneapolis (2), Montréal (2), Philadelphie (2), Toronto (2), Accra (1), Ahmedabad (1), Athènes (1), Atlanta (1), Birmingham (1), Boulder (1), Cape Town (1), Nanchang (1), Porto Alegre (1), Rome (1), Surate (1), Sydney (1), Vienne (1)}\\
        \hdashline
    \small{\textbf{\textcolor{blue}{Entre 1~000~000 et 3~000~000 d'habitant·e·s (33 références)}}}\\
\small{Rotterdam-La Haye (6), Amsterdam (4), Austin (3), Cincinnati (2), Cleveland (2), Copenhague (2), Oslo (2), Auckland (1), Bristol (1), Columbus (1), Helsinki (1), Gutenberg (1), Aix-Marseille-Provence (1), Nashville (1), Orlando (1), Poznań (1), Séville (1), Tucson (1), Turin (1)}\\
        \hdashline
    \small{\textbf{\textcolor{blue}{Entre 250~000 et 1~000~000 d'habitant·e·s (16 références)}}}\\
\small{Hamilton (3), Portland (3), Utrecht (3), Delft (2), Amstelland-Meerlanden (1), Eindhoven (1), Malmö (1), Mamelodi (1), Tarnow (1)}\\
        \hdashline
    \small{\textbf{\textcolor{blue}{Moins de 250~000 habitant·e·s (8 références)}}}\\
\small{Amboise (3), Bayeux (1), El Monte (1), Ithaca (1), Longmont (1), communes belges entre 30~000~et 200~000~habitant·e·s (1)}\\
        \hline
        \end{tabular}}
    \caption{Taille des agglomérations et des communes étudiées dans la revue systématique de la littérature.}
    \label{table-chap2:tailles-territoires-rsl}
        \vspace{5pt}
        \begin{flushleft}\scriptsize{
        \textcolor{blue}{Lecture~:} à partir de 198 études comprenant un terrain géographique, la revue systématique de la littérature est principalement composée d'agglomérations de plus de 3~000~000 d'habitant·e·s.
        }\end{flushleft}
        \begin{flushright}\scriptsize
        Auteur~: \textcolor{blue}{Dylan Moinse (2023)}
        \end{flushright}
        \end{table}