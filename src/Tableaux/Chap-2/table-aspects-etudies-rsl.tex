% Tableau aspects étudiés
%%Rédigé%%
        \begin{table}[h!]
    \centering
    \renewcommand{\arraystretch}{1.5}
    \resizebox{\columnwidth}{!}{
    \begin{tabular}{p{0.5\columnwidth}p{0.5\columnwidth}}
        %\hline
    \rule{0pt}{15pt} \small{\textbf{\textcolor{blue}{Aspects étudiés}}} & \small{\textbf{\textcolor{blue}{Sous-sections}}}\\
        \hline
    \multicolumn{2}{l}{\small{\textbf{\textcolor{blue}{Métadonnées et réseaux}}}}\\
\small{Auteur·rice·s, institutions, revues scientifiques, citations} & \small{\hyperref[chap2:etat-litterature-scientifique-internationale-btod]{sous-section 2.1} (page~\pageref{chap2:etat-litterature-scientifique-internationale-btod})}\\
    \hdashline
    \multicolumn{2}{l}{\small{\textbf{\textcolor{blue}{Terminologie}}}}\\
\small{Titre, mots-clés, résumé, contenu} & \small{\hyperref[chap2:analyse-textuelle]{sous-section 2.1.2} (page~\pageref{chap2:analyse-textuelle})}\\
    \hdashline
    \multicolumn{2}{l}{\small{\textbf{\textcolor{blue}{Objets d'étude}}}}\\
\small{mobilité individuelle légère, transports en commun, formes d'intégration intermodale} & \small{\hyperref[chap2:evolution-recherches-tc-mobilite-individuelle-legere]{sous-section 2.1.3} (page~\pageref{chap2:evolution-recherches-tc-mobilite-individuelle-legere})}\\
    \hdashline
    \multicolumn{2}{l}{\small{\textbf{\textcolor{blue}{Terrains géographiques}}}}\\
\small{Études de cas, échelles géographiques, comparaisons internationales} & \small{\hyperref[chap2:exploration-terrains-geographiques]{sous-section 2.1.4} (page~\pageref{chap2:exploration-terrains-geographiques})}\\
    \hdashline
    \multicolumn{2}{l}{\small{\textbf{\textcolor{blue}{Concepts}}}}\\
\small{Cadres théoriques, place du \acrshort{TOD}} & \small{\hyperref[chap2:fondements-theoriques]{sous-section 2.2.1} (page~\pageref{chap2:fondements-theoriques})}\\
    \hdashline
    \multicolumn{2}{l}{\small{\textbf{\textcolor{blue}{Méthodologie}}}}\\
\small{Méthodes de recherche, sources des données, échantillonnage, types d'analyse} & \small{\hyperref[chap2:methodes-collecte-donnees]{sous-sections 2.2.2} et \hyperref[chap2:demarches-types-analyses]{2.2.3} (pages \pageref{chap2:methodes-collecte-donnees} et \pageref{chap2:demarches-types-analyses})}\\
    \hdashline
    \multicolumn{2}{l}{\small{\textbf{\textcolor{blue}{Principes TOD (\Guillemets{\acrshort{7Ds}})}}}}\\
\small{Densité, diversité, conception, accessibilité des destinations, distance aux stations de transport en commun, gestion de la demande et inclusion sociale} & \small{\hyperref[chap2:densite-population]{sous-sections 3.1} (page~\pageref{chap2:densite-population}), \hyperref[chap2:diversite-fonctionnelle]{3.2} (page~\pageref{chap2:diversite-fonctionnelle}), \hyperref[chap2:traitement-espaces-publics]{3.3} (page~\pageref{chap2:traitement-espaces-publics}), \hyperref[chap2:accessibilite-intermodale]{3.4} (page~\pageref{chap2:accessibilite-intermodale}), \hyperref[chap2:distances-premiers-derniers-km]{3.5} (page~\pageref{chap2:distances-premiers-derniers-km}), \hyperref[chap2:gestion-demande-mobilite]{3.6} (page~\pageref{chap2:gestion-demande-mobilite}) et \hyperref[chap2:sociodemographie-usagers]{3.7} (page~\pageref{chap2:sociodemographie-usagers})}\\
    \hdashline
    \multicolumn{2}{l}{\small{\textbf{\textcolor{blue}{Comportements de mobilité}}}}\\
\small{Raisons, expérience, représentations sociales} & \small{\hyperref[chap2:comportements-mobilite]{sous-section 3.8} (page~\pageref{chap2:comportements-mobilite})}\\
    \hdashline
    \multicolumn{2}{l}{\small{\textbf{\textcolor{blue}{Impacts}}}}\\
\small{Mobilité, urbanisme, économie, environnement} & \small{\hyperref[chap2:impacts-systemes-urbain-mobilite]{sous-section 3.9} (page~\pageref{chap2:impacts-systemes-urbain-mobilite})}\\
        \hline
        \end{tabular}}
    \caption{Grille d'analyse de la revue systématique de la littérature sur un \textsl{Micromobility-friendly Transit-Oriented Development}.}
    \label{table-chap2:aspects-etudies-rsl}
        \vspace{5pt}
        \begin{flushleft}\scriptsize{
        \textcolor{blue}{Note~:} les aspects examinés ne se confinent pas à une seule thématique et apparaissent tout au long du chapitre.
        \\
        \textcolor{blue}{Lecture~:} la revue systématique de la littérature sur un urbanisme orienté vers les transports en commun et soutenu par la mobilité individuelle légère repose sur les métadonnées, la terminologie, les objets d'étude, les contextes géographiques, les concepts et les techniques mobilisés, les principes du modèle d'aménagement, les comportements de mobilité et les effets observés.
        }\end{flushleft}
        \begin{flushright}\scriptsize{
        Auteur~: \textcolor{blue}{Dylan Moinse (2023)}
        }\end{flushright}
        \end{table}