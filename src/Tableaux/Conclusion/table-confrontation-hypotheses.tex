% Tableau confrontation des hypothèses de recherche

    \begin{table}[h!]
    \centering
    \renewcommand{\arraystretch}{1.5}
    \resizebox{\columnwidth}{!}{
    \begin{tabular}{p{0.01\columnwidth}p{0.495\columnwidth}p{0.495\columnwidth}}
        %\hline
    \rule{0pt}{15pt} \small{\textbf{\textcolor{blue}{}}} & \small{\textbf{\textcolor{blue}{Résultats attendus}}} & \small{\textbf{\textcolor{blue}{Résultats observés}}}\\
        \hline
\cellcolor{green!20} & \textbf{Hypothèse 1} \small{(\hyperref[hypothese-1]{\(H_{1}\)}, page~\pageref{hypothese-1})} & \cellcolor{green!20}\textbf{\small{Corroborée}}\\
\cellcolor{green!20} & \small{\textsl{L’émergence de nouvelles thématiques de recherche sur le Transit-Oriented Development et la mobilité individuelle légère appelle une analyse conjointe.}} & \small{Si ces deux objets d’étude suscitent un intérêt croissant, ils sont encore trop souvent abordés séparément, alors que leur synergie offre des éclairages pertinents.}\\
    \hdashline
\cellcolor{orange!20} & \textbf{Hypothèse 2} \small{(\hyperref[hypothese-2]{\(H_{2}\)}, page~\pageref{hypothese-2})} & \cellcolor{orange!20}\textbf{\small{Partiellement invalidée}}\\
\cellcolor{orange!20} & \small{\textsl{Les recherches sur cette synergie intermodale restent conditionnées par l’association vélo et train et sont rarement mises en relation avec le concept d'aménagement.}} & \small{Le \textsl{Transit-Oriented Development} est souvent convoqué dans les études, bien qu'il soit inadéquatement exploité, et l’intégration des nouvelles solutions de mobilité devient un sujet d’étude croissant, hormis dans les contextes européens.}\\
    \hdashline
\cellcolor{orange!20} & \textbf{Hypothèse 3} \small{(\hyperref[hypothese-3]{\(H_{3}\)}, page~\pageref{hypothese-3})} & \cellcolor{orange!20}\textbf{\small{Partiellement invalidée}}\\
\cellcolor{orange!20} & \small{\textsl{La complexité des interactions entre réseau et territoire appelle une méthodologie systémique, capable de produire des résultats allant au-delà d'une simple juxtaposition.}} & \small{La complémentarité des approches a non seulement permis de générer de nouveaux questionnements, mais elle aussi su structurer une démarche multidimensionnelle et multiscalaire, non sans échapper à des risques de superposition et de contrainte temporelle.}\\
    \hdashline
\cellcolor{green!20} & \textbf{Hypothèse 4} \small{(\hyperref[hypothese-4]{\(H_{4}\)}, page~\pageref{hypothese-4})} & \cellcolor{green!20}\textbf{\small{Corroborée}}\\
\cellcolor{green!20} & \small{\textsl{Les pratiques intermodales, impliquant l'usage de la mobilité individuelle légère, s’intensifient sous l’effet de l'essor de la micro-mobilité, en dépit d'une appropriation inégalitaire par certains groupes sociaux.}} & \small{La mobilité individuelle légère, en tant que mode de transfert, connaît une \Guillemets{émergence} principalement portée par l'adoption de la trottinette électrique à usage personnel. Toutefois, son usage intermodal est plus inégalitaire qu'en monomodalité, bien que des interventions par l'urbanisme jouent un rôle modérateur sur les inégalités de genre.} \\
    \hdashline
\cellcolor{green!20} & \textbf{Hypothèse 5} \small{(\hyperref[hypothese-5]{\(H_{5}\)}, page~\pageref{hypothese-5})} & \cellcolor{green!20}\textbf{\small{Corroborée}}\\
\cellcolor{green!20} & \small{\textsl{Grâce à l’intégration de la mobilité individuelle légère, l’accessibilité multiscalaire par les transports en commun devient nettement plus performante, résiliente et compétitive.}} & \small{Les quartiers de gare accessibles à l'aide de la mobilité individuelle légère bénéficient d’une extension locale de leur périmètre et une augmentation des opportunités d’accès aux ressources régionales. Les choix d’itinéraire sont influencés par des stratégies de détour et de pause, qui, grâce à la portée et à la flexibilité offertes par ces véhicules, optimisent le déplacement intermodal.}\\
    \hdashline
\cellcolor{green!20} & \textbf{Hypothèse 6} \small{(\hyperref[hypothese-6]{\(H_{6}\)}, page~\pageref{hypothese-6})} & \cellcolor{green!20}\textbf{\small{Corroborée}}\\
\cellcolor{green!20} & \small{\textsl{Associer la mobilité individuelle légère aux stratégies d'aménagement représente une opportunité pour intégrer et étendre les quartiers de gare, mais aussi pour stimuler la fréquentation des transports en commun.}} & \small{Parmi les principaux leviers d’action en faveur d’un urbanisme orienté vers le rail, figure l’amélioration de la connectivité locale, avec un accent particulier sur les investissements dans le \Guillemets{système vélo} afin de dynamiser la fréquentation des gares, en appui aux orientations du \acrshort{TOD}.}\\
        \hline
        \end{tabular}}
    \caption{Test des hypothèses de recherche en fonction des conclusions de la recherche.}
    \label{table-conclusion:confrontation-hypotheses}
        \vspace{5pt}
        \begin{flushright}\scriptsize{
        Auteur~: \textcolor{blue}{Dylan Moinse (2025)}
        }\end{flushright}
        \end{table}