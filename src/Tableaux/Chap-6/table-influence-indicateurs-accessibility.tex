% Tableau Influence des indicateurs Accessibility
%%Rédigé%%
    \begin{table}[h!]
    \centering
    \renewcommand{\arraystretch}{1.5}
    \resizebox{\columnwidth}{!}{
    \begin{tabular}{p{0.08\columnwidth}p{0.38\columnwidth}p{0.18\columnwidth}p{0.18\columnwidth}p{0.18\columnwidth}}
        %\hline
    \rule{0pt}{15pt} \small{\textbf{\textcolor{blue}{ID}}} & \small{\textbf{\textcolor{blue}{Indicateur}}} & \small{\textbf{\textcolor{blue}{\(PI\)*}}} & \small{\textbf{\textcolor{blue}{\(CI\)*}}} & \small{\textbf{\textcolor{blue}{Perception}}}\\
        \hline
\small{\(A_{1}\)} & \small{Réseau piéton} & \small{\textbf{0,025} (7\textsuperscript{e})} & \small{\textbf{0,027} (8\textsuperscript{e})} & \underline{\small{\textbf{0,120}} (1\textsuperscript{er})}\\
\small{\(A_{2}\)} & \small{Densité d'intersection} & \small{\textbf{0,018} (8\textsuperscript{e})} & \small{\textbf{0,037} (7\textsuperscript{e})} & \small{\textbf{0,086} (7\textsuperscript{e})}\\
\small{\(A_{3}\)} & \small{Taux d'efficacité spatiale} & \small{\textbf{0,012} (9\textsuperscript{e})} & \small{\textbf{0,009} (10\textsuperscript{e})} & \small{\textbf{0,041} (11\textsuperscript{e})}\\
\small{\(A_{4}\)} & \small{Réseau cyclable} & \underline{\small{\textbf{0,110}} (4\textsuperscript{e})} & \underline{\small{\textbf{0,085}} (4\textsuperscript{e})} & \small{\textbf{0,086} (8\textsuperscript{e})}\\
\small{\(A_{5}\)} & \small{Stationnement vélo} & \underline{\small{\textbf{0,169}} (3\textsuperscript{e})} & \underline{\small{\textbf{0,171}} (3\textsuperscript{e})} & \underline{\small{\textbf{0,115}} (2\textsuperscript{e})}\\
\small{\(A_{6}\)} & \small{Services de \acrshort{VLS}} & \underline{\small{\textbf{0,239}} (2\textsuperscript{e})} & \underline{\small{\textbf{0,244}} (2\textsuperscript{e})} & \small{\textbf{0,084} (9\textsuperscript{e})}\\
\small{\(A_{7}\)} & \small{Services de métro et de tramway} & \underline{\small{\textbf{0,307}} (1\textsuperscript{er})} & \underline{\small{\textbf{0,284}} (1\textsuperscript{er})} & \underline{\small{\textbf{0,097}} (4\textsuperscript{e})}\\
\small{\(A_{8}\)} & \small{Services de \acrshort{BHNS} et de bus} & \small{\textbf{0,056} (5\textsuperscript{e})} & \small{\textbf{0,068} (5\textsuperscript{e})} & \small{\textbf{0,081} (10\textsuperscript{e})}\\
\small{\(A_{9}\)} & \small{Vitesse motorisée} & \small{\textbf{0,006} (10\textsuperscript{e})} & \small{\textbf{0,009} (9\textsuperscript{e})} & \underline{\small{\textbf{0,106}} (3\textsuperscript{e})}\\
\small{\(A_{10}\)} & \small{Stationnement automobile} & \small{\textbf{0,054} (6\textsuperscript{e})} & \small{\textbf{0,064} (6\textsuperscript{e})} & \small{\textbf{0,094} (5\textsuperscript{e})}\\
\small{\(A_{11}\)} & \small{Taux de motorisation} & \small{\textbf{0,004} (11\textsuperscript{e})} & \small{\textbf{0,003} (11\textsuperscript{e})} & \small{\textbf{0,088} (6\textsuperscript{e})}\\
        \hline
        \end{tabular}}
    \caption{Influence relative, par entropie, des indicateurs indépendants, liés à l'accessibilité locale (\(A\)), sur la fréquentation des gares.}
    \label{table-chap6:influence-indicateurs-accessibility}
        \vspace{5pt}
        \begin{flushleft}\scriptsize{
        \textcolor{blue}{Note~:} les statistiques \(PI\) et \(CI\) correspondent aux valeurs propres aux isochrones piétonnes et cyclables.
        \\
        \textcolor{blue}{Lecture~:} la desserte en métro et en tramway (\(A_{7}\)) explique statistiquement 30,7~\% et 28,4~\% de la fréquentation des gares, pour les périmètres piétons (\(PI\)) et cyclables (\(CI\)). Les effets attendus de cette variable s'élèvent seulement à 9,7~\% selon les aménageur·se·s.
        }\end{flushleft}
        \begin{flushright}\scriptsize{
        Réalisation~: \textcolor{blue}{Dylan Moinse (2024)}
        \\
        Auteur·rice·s~: projet de recherche \acrshort{NPART}
        }\end{flushright}
        \end{table}