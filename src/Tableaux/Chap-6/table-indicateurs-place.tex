% Tableau grille d'indicateurs (place)
%%Rédigé%%
    \begin{table}[h!]
    \centering
    \renewcommand{\arraystretch}{1.5}
    \resizebox{\columnwidth}{!}{
    \begin{tabular}{p{0.43\columnwidth}p{0.08\columnwidth}p{0.49\columnwidth}}
        %\hline
    \rule{0pt}{15pt} \small{\textbf{\textcolor{blue}{Catégorie}}} & \small{\textbf{\textcolor{blue}{ID}}} & \small{\textbf{\textcolor{blue}{Indicateur}}}\\
        \hline
    \multirow{2}{*}{\textbf{Densité}} & \small{\(P_{1}\)} & \small{Densité de population}\\
& \small{\(P_{2}\)} & \small{Densité d'emploi}\\
        \hdashline
    \multirow{5.5}{*}{\textbf{Diversité fonctionnelle}} & \small{\multirow{1.5}{*}{\(P_{3}\)}} & \small{Usage du sol à dominante résidentielle}\\
& \small{\multirow{1.5}{*}{\(P_{4}\)}} & \small{Usage du sol à dominante commerciale et dédiée aux services publics}\\
& \small{\multirow{1.5}{*}{\(P_{5}\)}} & \small{Usage du sol à dominante de bureaux et industrielle}\\
& \small{\multirow{1}{*}{\(P_{6}\)}} & \small{Usage du sol à dominante d'espaces verts}\\
        \hdashline
    \multirow{3}{*}{\textbf{Lieux d'attraction}} & \small{\(P_{7}\)} & \small{Points d'intérêt (\acrshort{POIs}) de \Guillemets{proximité}}\\
& \small{\multirow{1}{*}{\(P_{8}\)}} & \small{Points d'intérêt (\acrshort{POIs}) \Guillemets{intermédiaires}}\\
& \small{\(P_{9}\)} & \small{Points d'intérêt (\acrshort{POIs}) \Guillemets{supérieurs}}\\
        \hdashline
    \multirow{2.5}{*}{\textbf{Valeur du terrain}} & \small{\(P_{10}\)} & \small{Valeur foncière des biens résidentiels}\\
& \small{\multirow{1.5}{*}{\(P_{11}\)}} & \small{Valeur foncière des biens industriels, commerciaux et de bureaux}\\
        \hdashline
    \multirow{2}{*}{\textbf{Stratification sociale}} & \small{\(P_{12}\)} & \small{Proportion de logements sociaux}\\
& \small{\(P_{13}\)} & \small{Revenu moyen des ménages}\\
        \hline
        \end{tabular}}
    \caption{Grille d'indicateurs regroupés dans la dimension associée au \Guillemets{degré de développement urbain} (lieu).}
    \label{table-chap6:indicateurs-place}
        \vspace{5pt}
        \begin{flushleft}\scriptsize{
        \textcolor{blue}{Lecture~:} treize indicateurs sont regroupés dans la dimension \(P\). 
        }\end{flushleft}
        \begin{flushright}\scriptsize{
        Réalisation~: \textcolor{blue}{Dylan Moinse (2024)}
        \\
        Auteur·rice·s~: projet de recherche \acrshort{NPART}
        }\end{flushright}
        \end{table}