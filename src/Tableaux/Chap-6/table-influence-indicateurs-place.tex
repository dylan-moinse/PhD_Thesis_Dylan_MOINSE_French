% Tableau Influence des indicateurs Place
%%Rédigé%%
    \begin{table}[h!]
    \centering
    \renewcommand{\arraystretch}{1.5}
    \resizebox{\columnwidth}{!}{
    \begin{tabular}{p{0.08\columnwidth}p{0.38\columnwidth}p{0.18\columnwidth}p{0.18\columnwidth}p{0.18\columnwidth}}
        %\hline
    \rule{0pt}{15pt} \small{\textbf{\textcolor{blue}{ID}}} & \small{\textbf{\textcolor{blue}{Indicateur}}} & \small{\textbf{\textcolor{blue}{\(PI\)*}}} & \small{\textbf{\textcolor{blue}{\(CI\)*}}} & \small{\textbf{\textcolor{blue}{Perception}}}\\
        \hline
\small{\(P_{1}\)} & \small{Densité de population} & \small{\textbf{0,035} (10\textsuperscript{e})} & \small{\textbf{0,072} (7\textsuperscript{e})} & \small{\textbf{0,097} (7\textsuperscript{e})}\\
\small{\(P_{2}\)} & \small{Densité d'emploi} & \small{\textbf{0,068} (7\textsuperscript{e})} & \underline{\small{\textbf{0,104}} (4\textsuperscript{e})} & \small{\textbf{0,102} (6\textsuperscript{e})}\\
\small{\(P_{3}\)} & \small{Fonction résidentielle} & \small{\textbf{0,007} (13\textsuperscript{e})} & \small{\textbf{0,023} (12\textsuperscript{e})} & \small{\textbf{0,087} (11\textsuperscript{e})}\\
\small{\(P_{4}\)} & \small{Fonction commerciale} & \small{\textbf{0,039} (9\textsuperscript{e})} & \small{\textbf{0,062} (9\textsuperscript{e})} & \underline{\small{\textbf{0,119}} (1\textsuperscript{er})}\\
\small{\(P_{5}\)} & \small{Fonction industrielle et de bureaux} & \small{\textbf{0,070} (6\textsuperscript{e})} & \small{\textbf{0,062} (8\textsuperscript{e})} & \small{\textbf{0,089} (10\textsuperscript{e})}\\
\small{\(P_{6}\)} & \small{Présence d'espaces verts} & \underline{\small{\textbf{0,096}} (4\textsuperscript{e})} & \small{\textbf{0,094} (6\textsuperscript{e})} & \small{\textbf{0,105} (5\textsuperscript{e})}\\
\small{\(P_{7}\)} & \small{\acrshort{POIs} de \Guillemets{proximité}} & \small{\textbf{0,096} (5\textsuperscript{e})} & \small{\textbf{0,096} (5\textsuperscript{e})} & \underline{\small{\textbf{0,115}} (2-4\textsuperscript{e})}\\
\small{\(P_{8}\)} & \small{\acrshort{POIs} \Guillemets{intermédiaires}} & \underline{\small{\textbf{0,131}} (3\textsuperscript{e})} & \underline{\small{\textbf{0,118}} (3\textsuperscript{e})} & \underline{\small{\textbf{0,115}} (2-4\textsuperscript{e})}\\
\small{\(P_{9}\)} & \small{\acrshort{POIs} \Guillemets{supérieurs}} & \underline{\small{\textbf{0,160}} (2\textsuperscript{e})} & \underline{\small{\textbf{0,153}} (1\textsuperscript{er})} & \underline{\small{\textbf{0,115}} (2-4\textsuperscript{e})}\\
\small{\(P_{10}\)} & \small{Valeur des lieux résidentiels} & \small{\textbf{0,018} (11\textsuperscript{e})} & \small{\textbf{0,026} (11\textsuperscript{e})} & \small{\textbf{0,094} (8\textsuperscript{e})}\\
\small{\(P_{11}\)} & \small{Valeur des lieux d'activité} & \underline{\small{\textbf{0,206}} (1\textsuperscript{er})} & \underline{\small{\textbf{0,130}} (2\textsuperscript{e})} & \small{\textbf{0,012} (13\textsuperscript{e})}\\
\small{\(P_{12}\)} & \small{Part de logements sociaux} & \small{\textbf{0,055} (8\textsuperscript{e})} & \small{\textbf{0,046} (10\textsuperscript{e})} & \small{\textbf{0,092} (9\textsuperscript{e})}\\
\small{\(P_{13}\)} & \small{Revenu moyen des ménages} & \small{\textbf{0,018} (12\textsuperscript{e})} & \small{\textbf{0,016} (13\textsuperscript{e})} & \small{\textbf{0,087} (12\textsuperscript{e})}\\
        \hline
        \end{tabular}}
    \caption{Influence relative, par entropie, des indicateurs indépendants, liés à l'insertion urbaine (\(P\)), sur la fréquentation des gares.}
    \label{table-chap6:influence-indicateurs-place}
        \vspace{5pt}
        \begin{flushleft}\scriptsize{
        \textcolor{blue}{Note~:} les statistiques \(PI\) et \(CI\) correspondent aux valeurs propres aux isochrones piétonnes et cyclables.
        \\
        \textcolor{blue}{Lecture~:} la valeur foncière des lieux d'activité (\(P_{11}\)) explique statistiquement 20,6~\% et 13,0~\% de la fréquentation des gares, pour les périmètres piétons (\(PI\)) et cyclables (\(CI\)). Les effets attendus de cette variable s'élèvent seulement à 1,2~\% selon les aménageur·se·s.
        }\end{flushleft}
        \begin{flushright}\scriptsize{
        Réalisation~: \textcolor{blue}{Dylan Moinse (2024)}
        \\
        Auteur·rice·s~: projet de recherche \acrshort{NPART}
        }\end{flushright}
        \end{table}