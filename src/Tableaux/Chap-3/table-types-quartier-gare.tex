% % Tableau types de quartier de gare
% %%Rédigé%%
% \begin{table}[h!]
%   \centering
%   \renewcommand{\arraystretch}{1.5}
%     \resizebox{\columnwidth}{!}{%
%     \begin{tabular}{p{0.45\columnwidth}p{0.25\columnwidth}p{0.17\columnwidth}p{0.13\columnwidth}}
%          \hline
% %    \rule{0pt}{15pt} 
%     \small{\textcolor{blue}{\textbf{Hiérarchie des réseaux}}} 
%     & \small{\textcolor{blue}{\textbf{Taille}}} 
%     & \small{\textcolor{blue}{\textbf{Zone tampon}}} 
%     & \small{\textcolor{blue}{\textbf{Isochrone}}}
%       \\\hline
%       \small{Gare \acrshort{TER} et/ou \acrshort{TGV}}
%     & \multirow{2}{*}{\small{Accessibilité piétonne}}
%     & \multirow{2}{*}{\small{\(PB\)}}
%     & \multirow{2}{*}{\small{\(PI\)}}
%       \\
%       \small{Pôle d'échange multimodal}
%     &
%     &
%     &
%       \\ \hdashline
%       \small{Gare \acrshort{TER}}
%     & \multirow{2}{*}{\small{Accessibilité cyclable}}
%     & \multirow{2}{*}{\small{\(CB\)}}
%     & \multirow{2}{*}{\small{\(CI\)}}
%       \\
%       \small{Gare \acrshort{TGV} ou pôle d'échange multimodal}
%     &
%     &
%     & 
%     \\\hline
%   \end{tabular}
%   }
%     \caption{Typologie des quartiers de gare définis dans la région Hauts-de-France.}
%     \label{table-chap3:types-quartier-gare}
%     \vspace{5pt}
%     \begin{flushleft}\scriptsize{
%     \textcolor{blue}{Lecture~:} les zones tampons et les isochrones accessibles à pied ont une taille définie de douze minutes (équivalent à un kilomètre) autour des gares \acrshort{TER} et \acrshort{TGV} ainsi que des pôles d'échange multimodal. La même distance temporelle est fixée pour les zones tampons et les isochrones accessible en cycle à destination des gares et des haltes (équivalent à trois kilomètres), tandis que les quartiers de gare cyclables sont étendus à seize minutes (équivalent à quatre kilomètres) à vélo ou en micro-mobilité pour les pôles d'échange multimodal.
%     }\end{flushleft}
%     \begin{flushright}\scriptsize{%
%     Auteur~: \textcolor{blue}{Dylan Moinse (2022)}
%     }\end{flushright}
%   \end{table}
