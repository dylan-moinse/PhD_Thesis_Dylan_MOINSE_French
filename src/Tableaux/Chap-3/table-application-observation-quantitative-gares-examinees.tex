% Tableau Gares étudiées
%%Rédigé%%
  \begin{table}[h!]
    \centering
    \renewcommand{\arraystretch}{1.5}
    \resizebox{\columnwidth}{!}{
    \begin{tabular}{p{0.28\columnwidth}p{0.2\columnwidth}p{0.2\columnwidth}p{0.20\columnwidth}p{0.12\columnwidth}}
      % \hline
      \rule{0pt}{15pt} \small{\textcolor{blue}{\textbf{Gare ou halte}}} & \small{\textcolor{blue}{\textbf{Période (2022)}}} & \small{\textcolor{blue}{\textbf{Comptage*}}} & \small{\textcolor{blue}{\textbf{Trains}}} & \small{\textcolor{blue}{\textbf{Flux}}}\\
      \hline
    \multicolumn{5}{l}{\textbf{Gares de pôles régionaux}}\\
\small{Lille Flandres (\(G_1\))} & \small{5 et 7 avril} & \small{\textbf{5~836} (4,8~\%)} & \small{40 \acrshort{TGV} et \acrshort{TER}} & \small{21~992~946}\\
\small{Dunkerque (\(G_2\))} & \small{17 et 19 mai} & \small{\textbf{2~221} (21,2~\%)} & \small{28 \acrshort{TERGV} et \acrshort{TER}} & \small{1~905~250}\\
\small{Béthune (\(G_3\))} & \small{26 et 28 avril} & \small{\textbf{1~281} (14,6~\%)} & \small{13 \acrshort{TGV} et \acrshort{TER}} & \small{1~601~485}\\
\small{Armentières (\(G_4\))} & \small{3 et 5 mai} & \small{\textbf{2~324} (49,9~\%)} & \small{31 TER} & \small{850~773}\\
      \hdashline
\multicolumn{5}{l}{\textbf{Gare à rayonnement francilien}}\\
\small{Creil (\(G_5\))} & \small{7 et 9 juin} & \small{\textbf{2~159} (7,5~\%)} & \small{28 \acrshort{TER} et \acrshort{RER}} & \small{5~224~702}\\
      \hdashline
    \multicolumn{5}{l}{\textbf{Gares de rabattement sur les pôles régionaux}}\\
\small{Lille CHR (\(G_6\))} & \small{12 et 14 avril} & \small{\textbf{1~025} (59,9~\%)} & \small{42 \acrshort{TER}} & \small{312~323}\\
\small{Lesquin (\(G_7\))} & \small{19 et 21 avril} & \small{\textbf{309} (40,0~\%)} & \small{27 \acrshort{TER}} & \small{141~025}\\
\small{Le Poirier Université (\(G_8\))} & \small{10 et 12 mai} & \small{\textbf{280} (60,6~\%)} & \small{43 \acrshort{TER}} & \small{84~268}\\
\small{Vis-à-Marles (\(G_9\))} & \small{31 mai et 2 juin} & \small{\textbf{3} (3,2~\%)} & \small{2 \acrshort{TER}} & \small{17~063}\\
      \hdashline
    \multicolumn{5}{l}{\textbf{Échantillon complet}}\\
\multicolumn{2}{l}{\small{Neuf gares (18 jours, du 5 avril au 2 juin)}} & \small{\textbf{15~438} (8,5~\%)} & \small{254 trains} & \small{33~341~579}\\
      \hline
    \end{tabular}}
    \caption{Aperçu des neuf gares de la région Hauts-de-France formant le cadre géographique de l'observation quantitative.}
    \label{table-chap3:application-observation-quantitative-gares-examinees}
    \vspace{5pt}
        \begin{flushleft}\scriptsize
        \textcolor{blue}{Note~:} le comptage se réfère à l'échantillon récolté pour chacune des gares. Les proportions représentent quant à elles la part de voyageur·se·s observé·e·s par rapport à leur fréquentation quotidienne en 2022.
        \\
        \textcolor{blue}{Lecture~:} parmi les neuf gares ayant fait l'objet de séances d'observation quantitative au cours d'un mardi et d'un jeudi, nous avons pu capturer environ 8,5~\% des flux quotidiens.
        \end{flushleft}
        \begin{flushright}\scriptsize
        Jeux de données~: \textsl{SNCF Open Data} \textcolor{blue}{\autocite{sncf_frequentation_2024}}
        \\
        Auteur~: \textcolor{blue}{Dylan Moinse (2023)}
        \end{flushright}
        \end{table}