% Gares étudiées
%%Rédigé%%
        \begin{table}[h!]
  \centering
  \renewcommand{\arraystretch}{1.5}
  \resizebox{\columnwidth}{!}{
  \begin{tabular}{p{0.2\columnwidth}p{0.48\columnwidth}p{0.15\columnwidth}p{0.17\columnwidth}}
    % \hline
    \rule{0pt}{15pt} \small{\textcolor{blue}{\textbf{Gare ou halte}}} & \small{\textcolor{blue}{\textbf{\acrshort{EPCI}}}} & \small{\textcolor{blue}{\textbf{Département}}} & \small{\textcolor{blue}{\textbf{Vélo* (2020)}}}\\
        \hline
    \multicolumn{4}{l}{\small{\textbf{Gare Lille Flandres} (\(G_1\))}}\\
\small{Lille} & \small{\acrfull{MEL}} & \small{Nord} & \small{6,6~\% (3,8~\%)}\\
        \hdashline
    \multicolumn{4}{l}{\small{\textbf{Gare de Dunkerque} (\(G_2\))}}\\
\small{Dunkerque} & \small{\acrfull{CUD}} & \small{Nord} & \small{3,5~\% (3,2~\%)}\\
        \hdashline
    \multicolumn{4}{l}{\small{\textbf{Gare de Béthune} (\(G_3\))}}\\
\multirow{1.5}{*}{\small{Béthune}} & \small{\acrfull{CABBALR}} & \multirow{1.5}{*}{\small{Pas-de-Calais}} & \multirow{1.5}{*}{\small{2,9~\% (1,4~\%)}}\\
        \hdashline
    \multicolumn{4}{l}{\small{\textbf{Gare d'Armentières} (\(G_4\))}}\\
\small{Armentières} & \small{\acrfull{MEL}} & \small{Nord} & \small{3,4~\% (3,8~\%)}\\
        \hdashline
    \multicolumn{4}{l}{\small{\textbf{Gare de Creil} (\(G_5\))}}\\
\small{Creil} & \small{\acrfull{ACSO}} & \small{Oise} & \small{1,1~\% (1,1~\%)}\\
        \hdashline
    \multicolumn{4}{l}{\small{\textbf{Halte Lille CHR} (\(G_6\))}}\\
\small{Lille} & \small{\acrfull{MEL}} & \small{Nord} & \small{6,6~\% (3,8~\%)}\\
        \hdashline
    \multicolumn{4}{l}{\small{\textbf{Gare de Lesquin} (\(G_7\))}}\\
\small{Lesquin} & \small{\acrfull{MEL}} & \small{Nord} & \small{2,2~\% (3,8~\%)}\\
        \hdashline
    \multicolumn{4}{l}{\small{\textbf{Halte Le Poirier Université} (\(G_8\))}}\\
\multirow{1.5}{*}{\small{Trith-Saint-Léger}} & \small{\acrfull{Porte du Hainaut}} & \multirow{1.5}{*}{\small{Nord}} & \multirow{1.5}{*}{\small{1,0~\% (1,8~\%)}}\\
        \hdashline
    \multicolumn{4}{l}{\small{\textbf{Halte Vis-à-Marles} (\(G_9\))}}\\
\multirow{1.5}{*}{\small{Marles-les-Mines}} & \small{\acrfull{CABBALR}} & \multirow{1.5}{*}{\small{Pas-de-Calais}} & \multirow{1.5}{*}{\small{0,9~\% (1,4~\%)}}\\
        \hline
        \end{tabular}}
    \caption{Répartition modale de l'usage monomodal du vélo dans les territoires d'implantation.}
    \label{table-chap3:part-modale-velo-gares-examinees}
        \vspace{5pt}
        \begin{flushleft}\scriptsize{
        \textcolor{blue}{Note~:} la dernière colonne du tableau se réfère à la part modale du vélo à l'échelle de la commune puis à celle de l'\acrshort{EPCI} entre parenthèses.
        \\
        \textcolor{blue}{Lecture~:} parmi les neuf gares examinées, les déplacements exclusifs à vélo sont bien plus développés à Lille, suivent la moyenne nationale à Dunkerque, à Armentières et à Béthune, et sont marginaux dans les communes de Lesquin, de Creil, de Trith-Saint-Léger et de Marles-les-Mines.
        }\end{flushleft}
        \begin{flushright}\scriptsize{
        Jeux de données liés à la part modale communale et intercommunale du vélo~: \textsl{Atlas vélo régional Hauts-de-France} de \textcolor{blue}{\textcite{velo__territoires_atlas_2023}}, elles-mêmes issues du fichier \textsl{Mobilités Professionnelles (MOBPro) du recensement de la population de 2020} de l'\textcolor{blue}{\textcite{insee_documentation_2023}}
        \\
        Jeux de données liés à l'offre de services vélo~: \textcolor{blue}{\textcite{sncf_voyageurs_stationnement_2023}}, \textcolor{blue}{\textcite{ilevia_abris_nodate}} et \textcolor{blue}{\textcite{openstreetmap_openstreetmap_2023}} 
        \\
        Auteur~: \textcolor{blue}{Dylan Moinse (2023)}
        }\end{flushright}
        \end{table}